\section{Modelos de Casos de Uso}
\subsection{Modelo de Casos de Uso de alto nível}

Diagrama geral de Use Cases

\begin{comment}
\begin{figure}[!htb]
	\centering
	\includegraphics[scale=0.85]{images/diagramaGeralDeUseCase}
	\caption {Diagrama geral de \emph{Use Cases} do produto}
\end{figure}
\end{comment}


\subsection{Modelos de Casos de Uso detalhado}


\subsubsection{\textbf{1 - HoneyPot}}

Tal como apresentado no diagrama abaixo, o QEMU é capaz de identificar system calls do guest OS mais seu contexto. Após detectar uma chamada este
actualiza a Base de dados com essa informação. O QEMU pode também copiar qualquer ficheiro do guest OS para o Host OS ou para outro ponto remoto.
Mais ainda o QEMU é capaz de capturar uma sequência de capturas de ecrã do que se passa no guest, criando assim um ficheiro em formato vídeo para, 
mais tarde o utilizador poder ver.

\begin{comment}
Qualquer utilizador do produto poderá efectuar uma simulação de seguro automóvel. No entanto, cada utlizador terá acesso a pequenas funcionalidades diferentes dependendo do tipo de utiliador em que se insere (registado ou não, mediador, etc). Aqui apresentaremos o fluxo normal deste \emph{use case} para um utilizador básico, ou seja, sem qualquer tipo de privilégios. Mais à frente, na explicação dos \emph{use cases} que se seguem, abordaremos as funcionalidades extra que cada tipo de utilizador terá. 
\end{comment}

\begin{figure}[!htb]
	\centering
	\includegraphics[scale=0.80]{images/ucs/HoneyPot}
	\caption {Diagrama de use case, parte do HoneyPot}
\end{figure}
\pagebreak




\subsubsection{\textbf{2 - Rede}}

O Delusion Collector Daemon (DCD) tem como função capturar todo o tipo de trâfego que passe pela máquina. Para isso acontecer, o DCD usa o Snort e o
Tshark (duas ferramentas de redes), para captura de trâfego e geração de alertas. Quando o DCD recolhe a informação referida, guarda-a numa 
base de dados específica. Podemos assim ver o DCD como um coletor de toda a informação que tanto o Snort como o TShark geram, passando de seguida
essa informação para uma bd.

\begin{comment}
A realização de seguro de saúde é semelhante à automóvel, sendo que os dados a serem fornecidos são diferentes. De momento, estamos a considerar relevantes apenas o sexo e data de nascimento das pessoas seguradas, assim como o grau de parentesco entre si. Temos portanto em conta que podem ser seguradas várias pessoas e que o tomador não é necessariamente uma delas.
\end{comment}


\begin{figure}[!htb]
	\centering
	\includegraphics[scale=0.80]{images/ucs/Rede}
	\caption {Diagrama de use case, parte da Rede}
\end{figure}
\pagebreak

\newcommand{\uticomum}{\emph{utilizador comum}\xspace} 
\newcommand{\admini}{\emph{administrador}\xspace} 
\newcommand{\visualz}{\emph{visualizador}\xspace}
\subsubsection{\textbf{3 - Visualizador}}

Como já foi referido anteriormente (sec x) existem dois tipos de utilizadores: o \uticomum; e o \admini.\\ 

O \admini terá a mesma interacção com o \visualz que o \uticomum tem (que será explicado mais adiante), mais uma responsabilidade acrescida, 
conforme se pode ver na Figura~\ref{fig: casodeusovisual}:

\begin{itemize}
 \item \textbf{Configurar utilizadores} - esta funcionalidade diz respeito à gestão de utilizadores e engloba acções do tipo adicionar/remover utilizadores, 
 alterar permissões (por exemplo: acesso a configurações, tipos de visualizações permitidas, etc).
\end{itemize}

O \uticomum terá como interacção com o sistema as seguintes funcionalidades:

\begin{itemize}
 \item \textbf{Ver Tráfego} - a funcionalidade que diz respeito a todas as formas de visualização que serão disponibilizadas pelo \visualz. Maioritariamente, a informação será exibida através de gráficos (de diversos tipos) e tabelas.
 \item \textbf{Ver Alertas} - os alertas que serão processados pelo \visualz, tanto poderão dizer respeito ao \emph{honeypot} como à rede.
 \item \textbf{Aplicar Filtros} - o utilizador poderá aplicar filtros às suas pesquisas, de modo a focalizar a informação ao que realmente lhe interessa.
 \item \textbf{Configurar Rede} - esta funcionalidade diz respeito às configurações que o utilizador poderá fazer à rede. Consoante a sua permissão, poderá por exemplo configurar um \emph{router} ou servidor.
 \item \textbf{Configurar Honeypot} - Finalmente, o utilizador poderá também configurar o \emph{honeypot}, alterando certos parâmetros, como por exemplo: tamanho de memória RAM, Sistema Operativo, etc.
\end{itemize}
\begin{comment}
É possível incluir vários produtos numa mesma simulação. Para o efeito, o utilizador procederá como numa simulação normal, preenchendo os dados correspondentes ao primeiro produto a simular. Terminada esta simulação, é apresentada ao utilizador a opção "acrescentar novo produto (simulação multi-produto)" que, quando seleccionada, agrega a presente simulação à carteira de simulações e apresenta a possibilidade de realização de simulação de um novo produto. 

Repetindo este processo, e quando satisfeito com o número de produtos simulados, o Utilizador solicita o cálculo do prémio conjunto, que tem em conta os descontos aplicáveis a uma simulação multi-produto.
\end{comment}


\begin{figure}[!htb]
	\centering
	\includegraphics[scale=0.80]{images/ucs/Visualizador}
	\caption {Diagrama de use case, parte da Rede}
\end{figure}
\pagebreak






\subsubsection{\textbf{4 - Configuração HoneyPot}}

Detalhando mais os tipos de configurações que um utilizador pode fazer. Tal como já foi dito, este pode indicar o SO que quer que a máquina corra, 
pode também indicar os serviços activos e suas respectivas vulnerabilidades. Importante referir que na escolha do SO, o utilizador pode indicar os 
parâmetros de hardware da máquina.

\begin{comment}
A cada utilizador registado com simulações guardadas será dada a possibilidade de pesquisar pelas mesmas. Tal significa filtrar as simulações guardadas por produto, valor do prémio, etc. O mediador poderá também pesquisar as simulações dos seus clientes. O sistema permite também a consulta das simulações pesquisadas.
\end{comment}

\begin{figure}[!htb]
	\centering
	\includegraphics[scale=0.80]{images/ucs/ConfHoneyPot}
	\caption {Diagrama de use case, configuração HoneyPot}
\end{figure}
\pagebreak

\subsubsection{\textbf{5 - Configuração Rede}}

Mais uma vez, detalhando o que o utilizador pode configurar na rede. Este pode configurar a firewall (Iptables), tal como portas a bloquear, ou gamas
de ip's a bloquear. Pode configurar o Snort e o TShark para o tipo de capturas de trâfego e gerações de alertas pretendidos.

\begin{comment}
A cada utilizador registado com simulações guardadas será dada a possibilidade de pesquisar pelas mesmas. Tal significa filtrar as simulações guardadas por produto, valor do prémio, etc. O mediador poderá também pesquisar as simulações dos seus clientes. O sistema permite também a consulta das simulações pesquisadas.
\end{comment}

\begin{figure}[!htb]
	\centering
	\includegraphics[scale=0.80]{images/ucs/ConfRede}
	\caption {Diagrama de use case, configuração Rede}
\end{figure}
\pagebreak

\subsubsection{\textbf{6 - Configuração Utilizadores}}

Tal como já foi dito, o administrador e só o administrador pode criar e remover utilizadores, bem como, gerir o seu perfil. Dentro do perfil este
pode configurar as permissões de cada utilizador, password de acesso entre outros.

\begin{comment}
A cada utilizador registado com simulações guardadas será dada a possibilidade de pesquisar pelas mesmas. Tal significa filtrar as simulações guardadas por produto, valor do prémio, etc. O mediador poderá também pesquisar as simulações dos seus clientes. O sistema permite também a consulta das simulações pesquisadas.
\end{comment}

\begin{figure}[!htb]
	\centering
	\includegraphics[scale=0.80]{images/ucs/ConfUtilizadores}
	\caption {Diagrama de use case, configuração Utilizadores}
\end{figure}
\pagebreak

