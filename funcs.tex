\pagebreak
\section{Requisitos Funcionais}
De seguida apresentamos todos os requisitos funcionais do projecto. Apresentamos numa escala de 1 a 5 o nivel de prioridade, sendo 1 o mais baixo e 5 o mais alto.

\begin{minipage}{0.55\textwidth}
\begin{flushleft}\textbf{Requirement \#: 1}\end{flushleft}
\end{minipage}
\begin{minipage}{0.4\textwidth}
\end{minipage}

\begin{description}
\item \textbf{Description}:

O Qemu deve poder identificar chamadas do sistema e seu contexto, de forma automática. \\

\item \textbf{Rationale}:

Uma vez que o utilizador malicioso irá usar a máquina virtual, para seu proveito, é imperativo que seja registado
tudo o que ele faça no sistema. Umas das formas de registo será esta de identificar system calls e o contexto em que são chamadas,
nomeadamente id do processo, id do utilizador, e argumentos das system calls.

\item \textbf{Fit Criterion}:

\textbf{Todas} as system calls chamadas pelo guest OS, bem como, respectivos pid,uid e argumentos tem que ser registadas numa tabela. 

\item \textbf{Priority}:

5 em 5

\end{description}

\pagebreak


% Fim do requisito 









% Início do requisito 


\begin{minipage}{0.55\textwidth}
\begin{flushleft}\textbf{Requirement \#: 2}\end{flushleft}
\end{minipage}
\begin{minipage}{0.4\textwidth}
\end{minipage}

\begin{description}
\item \textbf{Description}:

O Qemu deve poder filmar as acções do utilizador malicioso, disponibilizando de seguida, o ficheiro respectivo em formato vídeo.\\


\item \textbf{Rationale}:

Como mais valia para o projecto, e como alternativa ao modo de visualização estático, o utilizador poderá ver de forma, mais "leve" o que
se passa no guest OS, proporciando-se, assim, uma maior maior eficiência na análise do ocorrido.


\item \textbf{Fit Criterion}:

Tem que ser possível ver um filme (sequência de capturas de ecrã) do guest OS, através de um ficheiro em formato vídeo,
para cada sessão que um utilizador malicioso tenha iniciado.

\item \textbf{Priority}:

2 em 5

\end{description}

\pagebreak


% Fim do requisito 


% Início do requisito 


\begin{minipage}{0.55\textwidth}
\begin{flushleft}\textbf{Requirement \#: 3}\end{flushleft}
\end{minipage}
\begin{minipage}{0.4\textwidth}
\end{minipage}

\begin{description}
\item \textbf{Description}:

O Qemu pode copiar qualquer ficheiro, que tenha sido interagido pelo utilizador malicioso, para o sistema onde está em funcionamento, 
ou, para um sistema remoto.

\item \textbf{Rationale}:

Por si só o registo de system calls é suficiente para registar os eventos iniciados pelo utilizador malicioso. Poderá no entanto haver casos
em que este faz um download de um ficheiro malicioso (onde detectar system calls pouco nos ajuda para analisar o que está presente no ficheiro), ou 
altera/cria ficheiros de texto. Nestes casos seria muito útil poder fazer uma cópia desses ficheiros para outro sistema para uma posterior análise.

\item \textbf{Fit Criterion}:

O Qemu ser capaz de transferir qualquer ficheiro para o Host ou mesmo para um sistema remoto.

\item \textbf{Priority}:

4 em 5

\end{description}

\pagebreak


% Fim do requisito 



% Início do requisito 


\begin{minipage}{0.55\textwidth}
\begin{flushleft}\textbf{Requirement \#: 4}\end{flushleft}
\end{minipage}
\begin{minipage}{0.4\textwidth}
\end{minipage}

\begin{description}
\item \textbf{Description}:

O daemon collector do Delusion (DCD), deve poder capturar qualquer tipo de trâfego que passe pela máquina virtual.

\item \textbf{Rationale}:

Até agora só foi falado de fazer registos de eventos após sessão iniciada pelo utilizador malicioso. No entanto, é muito importante
que o sistema seja capaz de registar todos os eventos a partir de que o utilizador tente interagir com a máquina virtual. Coisa que não é por si 
só possível com os requisitos anteriores.

\item \textbf{Fit Criterion}:

\textbf{Todo} o tráfego que passa pela máquina virtual tem que ser passível de ser registado, seja no sistema local ou num remoto.

\item \textbf{Priority}:

5 em 5

\end{description}

\pagebreak


% Fim do requisito 

% Início do requisito 


\begin{minipage}{0.55\textwidth}
\begin{flushleft}\textbf{Requirement \#: 5}\end{flushleft}
\end{minipage}
\begin{minipage}{0.4\textwidth}
\end{minipage}

\begin{description}
\item \textbf{Description}:

O utilizador consegue ver alertas, gerados pelo DCD, num browser.

\item \textbf{Rationale}:

O requisito anterior apesar de capturar toda a informação é demasiado verbosa para o utilizador. Se o sistema conseguir detectar padrões no 
trafego e avisar o utilizador desses padrões a detecção de eventos malignos será muito mais eficiente.

\item \textbf{Fit Criterion}:

Para um conjunto de tipo de ataques previamente definidos pelos utilizador. O sistema tem que ser capaz de conseguir detectar o padrão respectivo
a cada um deles e gerar um alerta.

\item \textbf{Priority}:

4 em 5

\end{description}

\pagebreak


% Fim do requisito 

% Início do requisito 


\begin{minipage}{0.55\textwidth}
\begin{flushleft}\textbf{Requirement \#: 6}\end{flushleft}
\end{minipage}
\begin{minipage}{0.4\textwidth}
\end{minipage}

\begin{description}
\item \textbf{Description}:

O utilizador consegue ver o trâfego que passou na máquina, capturado pelo DCD, num browser.

\item \textbf{Rationale}:
Para efeitos de simplicidade é extremamente útil que toda a informação respectiva ao trâfego convirga, de modo que, o utilizador 
não tenha que andar a analisar vários ficheiros distribuídos pelo sistema.

\item \textbf{Fit Criterion}:
Toda o trâfego coletado pelo sistema,tem que estar presente num ponto comum.

\item \textbf{Priority}:

4 em 5

\end{description}

\pagebreak


% Fim do requisito 


% Início do requisito 


\begin{minipage}{0.55\textwidth}
\begin{flushleft}\textbf{Requirement \#: 7}\end{flushleft}
\end{minipage}
\begin{minipage}{0.4\textwidth}
\end{minipage}

\begin{description}
\item \textbf{Description}:

O utilizador configurar os parâmetros de rede do HoneyPot, através dum browser.

\item \textbf{Rationale}:

Mais uma vez as ferramentas de configuração de redes, devem convergir numa só, de modo a que o utlizador não tenha que usar várias ferramentas
ao mesmo tempo, promovendo assim mais uma vez a simplicidade

\item \textbf{Fit Criterion}:

As ferramentas necessárias para configuração de rede do HoneyPot, tem que estar disponíveis a partir do visualizador.

\item \textbf{Priority}:

4 em 5

\end{description}

\pagebreak


% Fim do requisito 


% Início do requisito 


\begin{minipage}{0.55\textwidth}
\begin{flushleft}\textbf{Requirement \#: 8}\end{flushleft}
\end{minipage}
\begin{minipage}{0.4\textwidth}
\end{minipage}

\begin{description}
\item \textbf{Description}:

O utlizador consegue filtrar a informação, referente a eventos, através dum browser.

\item \textbf{Rationale}:

É importante que a informação consiga ser filtrada pelo utilizador, já que, este permite um maior foco na parte que o utilizador está
interessado, aumentando assim a produtividade da análise pretendida.

\item \textbf{Fit Criterion}:

A informação tem que ser possível de ser filtrada pelo visualizador, através de critérios definidos pelo utilizador.

\item \textbf{Priority}:

4 em 5

\end{description}

\pagebreak


% Fim do requisito 


% Início do requisito 


\begin{minipage}{0.55\textwidth}
\begin{flushleft}\textbf{Requirement \#: 9}\end{flushleft}
\end{minipage}
\begin{minipage}{0.4\textwidth}
\end{minipage}

\begin{description}
\item \textbf{Description}:

O utlizador consegue, escolher o tipo de gráfico usado para representar o trâfego e alertas, através dum browser.

\item \textbf{Rationale}:

Um conjunto de gráficos à escolha permite ao utilizador escolher o gráfico que mais se adequa ao tipo de informação e ao tipo de análise feita,
originando-se assim uma representação mais adequada à informação presente.


\item \textbf{Fit Criterion}:

A informação tem que ser possível de ser representada a partir de um gráfico previamente escolhida pelo utilizador.

\item \textbf{Priority}:

4 em 5

\end{description}

\pagebreak


% Fim do requisito 


% Início do requisito 


\begin{minipage}{0.55\textwidth}
\begin{flushleft}\textbf{Requirement \#: 10}\end{flushleft}
\end{minipage}
\begin{minipage}{0.4\textwidth}
\end{minipage}

\begin{description}
\item \textbf{Description}:

O utilizador consegue, configurar uma ou mais instâncias do HoneyPot, através dum browser.

\item \textbf{Rationale}:

No projecto, não é nosso interesse que o tipo de vulnerabilidade do sistema seja sempre o mesmo. É sim do nosso interesse que possamos criar
máquinas com vulnerabilidades diferentes permitindo assim uma melhor selecção dos utilizadores maliciosos a usar a máquina. Também é do nosso interesse
criar máquinas com características diferentes, por exemplo: arquitecturas, periféricos, memória RAM, etc\ldots Com isto cada máquina poderá ser adaptada
a cada situação pretendida.

\item \textbf{Fit Criterion}:

Tem que ser possível configurar os vários parâmetros de instâncias do HoneyPot, através de um browser

\item \textbf{Priority}:

4 em 5

\end{description}

\pagebreak


% Fim do requisito 


% Início do requisito 


\begin{minipage}{0.55\textwidth}
\begin{flushleft}\textbf{Requirement \#: 11}\end{flushleft}
\end{minipage}
\begin{minipage}{0.4\textwidth}
\end{minipage}

\begin{description}
\item \textbf{Description}:

O administrador consegue criar e remover utilizadores, através dum browser.


\item \textbf{Rationale}:
Não queremos que os utilizadores que podem aceder ao visualizador sejam fixos. Pode existir um futuro (muito provavelmente) onde irá ser preciso
uma pessoa para monitorar o HoneyPot, ou então seja preciso remover uma pessoa dessa monitorização. Posto isto será muito importante poder adicionar
ou remover utilizadores conforme necessário.

\item \textbf{Fit Criterion}:

O administrador poder criar ou remover utilizadores.

\item \textbf{Priority}:

5 em 5

\end{description}

\pagebreak


% Fim do requisito 


% Início do requisito 


\begin{minipage}{0.55\textwidth}
\begin{flushleft}\textbf{Requirement \#: 12}\end{flushleft}
\end{minipage}
\begin{minipage}{0.4\textwidth}
\end{minipage}

\begin{description}
\item \textbf{Description}:

O administrador gerir as permissões de utilizadores, através dum browser.


\item \textbf{Rationale}:

Como todos os utilizadores não são os mesmos, podemos querer dar acessos a algumas partes do HoneyPot a uns e negar o acesso a outros, para que isso aconteca é imperativo que haja um sistema de permissões que permita dar níveis de permissão a cada utilizador.

\item \textbf{Fit Criterion}:

O administrador tem que conseguir alterar as permissões de cada utilizador.

\item \textbf{Priority}:

4 em 5

\end{description}

\pagebreak


% Fim do requisito 

