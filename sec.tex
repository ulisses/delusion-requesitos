\section{Requisitos de segurança}
Neste item iremos esclarecer alguns pontos relacionados com as questões de acesso, autorização, permissões e privacidade nos acessos ao sistema.
\textbf{Acesso} As questões a nível de acesso físico ao servidor terão que ser tratadas com a empresa tomadora do produto e analisadas caso a caso,
consoante os recursos humanos que a empresa terá para lidar com a sua infra-estrutura tecnológica.
Contudo, podemos dizer que o acesso físico ao servidor deverá apenas ser feito, ou autorizado, pelo director dos sistemas de
informação ou pelo responsável do departamento informático, podendo incluir:

\begin{itemize}
\item Administrador do sistema
\item Técnicos, nomeadamente para manutenção
\end{itemize}

Relativamente aos dados em si, existem algumas restrições no acesso e visualização dos mesmos, nomeadamente:
\begin{itemize}
\item Cada utilizador só deverá ter acesso à porção da rede que o Administrador delegar em si.
\item O administrador do sistema terá acesso à globalidade da informação.
\end{itemize}

\textbf{Integridade}
O bom funcionamento do sistema está dependente da performance a nível aplicacional e a nível do sistema de informação.
Assim, são enumeradas algumas medidas a ter em consideração para assegurar esse bom funcionamento:
\begin{itemize}
\item Integridade da rede e comunicações. Este produto não assegura a prevenção de ataques de terceiros à rede.
\item Incluir a nova aplicação no sistema de backups da empresa, nomeadamente a base de dados.
\end{itemize}

A nível de funcionamento aplicacional serão também tidos alguns cuidados de modo a garantir segurança para quem tem acesso aos dados colectados.

\textbf{Privacidade}
O sistema terá que garantir segurança na ligação ao browser onde irão aparecer os dados do Visualizador e apenas
permitir o acesso a pessoas registadas no sistema do Visualizador.

\subsection{Requisitos políticos}
Esta secção contém requisitos específicos relacionados com factores políticos que afectam a aceitabilidade do produto.
\begin{itemize}
\item O produto deverá fornecer todas as funcionalidades ao administrador da empresa (CEO), se este o pretender.
\end{itemize}

\section{Requisitos Legais}
\subsection{Requisitos de conformidade}
Esta secção refere os requisitos legais do produto.
\begin{itemize}
\item O produto deverá ir de encontro à legislação em vigor no país.
\item O produto deverá respeitar a propriedade intelectual bem como os direitos de autor.
\item O produto deverá armazenar os dados segundo as normas e directrizes dadas pelo sistema nacional de protecção de dados.
\end{itemize}

