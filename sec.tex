\section{Requisitos de segurança}
Neste item iremos esclarecer alguns pontos relacionados com as questões de acesso, autorização, permissões e privacidade nos acessos aos dados do sistema. 

\textbf{Acesso}
As questões a nível de acesso físico ao servidor terão que ser tratadas com a empresa tomadora do produto e analisados caso a caso consoante também os recursos humanos que a empresa terá para lidar com a sua infra-estrutura tecnológica.
Contudo, genericamente, podemos dizer que o acesso físico ao servidor deverá somente ser feito ou autorizado pelo director dos sistemas de informação ou responsável pelo departamento informático, podendo incluir, nomeadamente:

\begin{itemize}
\item Administrador do sistema
\item Técnicos, nomeadamente para manutenção
\end{itemize}

Relativamente aos dados em si, existem algumas restrições no acesso e visualização dos mesmos, nomeadamente:
\begin{itemize}
\item Cada cliente só deverá ter acesso à sua própria informação
\item Cada mediador só deverá ver os dados dos seus clientes
\item O colaborador gere somente a configuração de produtos e fórmulas
\item O administrador do sistema terá acesso à globalidade da informação
\end{itemize}

\textbf{Integridade}
O bom funcionamento do sistema está dependente do bom funcionamento deste a nível aplicacional e a nível do sistema de informação. Assim são enumeradas algumas medidas a levar em consideração para assegurar esse bom funcionamento:
\begin{itemize}
\item Integridade da rede e comunicações, evitando assim acessos não autorizados
\item Incluir, se possível alguma redundância a nível de servidores de modo a aumentar a segurança da disponibilidade do serviço.
\item Incluir a nova aplicação no sistema de backups da empresa, nomeadamente a base de dados.
\end {itemize}

A nível de funcionamento aplicacional serão também tidos alguns cuidados de modo a aumentar a coerência dos dados inseridos, designadamente a validação dos dados recolhidos pela aplicação para que estes estejam de acordo com o modelo do sistema de informação.

\textbf{Privacidade}
O sistema informará o cliente que todos os dados armazenados serão confidenciais e não disponibilizados a terceiros para outro tipo de tratamento que não o inerente ao funcionamento do sistema em si.
Qualquer alteração nesta politica de funcionamento será atempadamente comunicada ao cliente e terá que ter a concordância deste.
Os dados armazenados seguirão as normas e directrizes dadas pelo sistema nacional de protecção de dados bem.

\textbf{Auditoria}
O sistema operativo poderá guardar informação relevante para uma auditoria ao funcionamento da aplicação, no entanto a nível interno, o sistema poderá guardar histórico de informação inserida, alterada ou removida para facilitar um processo de auditoria ao funcionamento da aplicação.

\textbf{Imunização}
A aplicação não prevê qualquer tipo de protecção adicional no que diz respeito a possíveis ataques de programas não autorizados ou indesejados, como sendo infecções de sistema por viroses, cavalos de tróia entre outros. A protecção é dada pelos acessos autorizados dos utilizadores à aplicação, quaisquer outras considerações terão que ser previstas no conjunto de protecções necessárias às máquinas que alojam os sistemas aplicionais e de informação e à própria rede em si.

\section{Requisitos culturais e políticos}
\subsection{Requisitos culturais}
Nesta secção serão apresentação os requisitos específicos relacionados com os factores sociológicos que afectam a aceitabilidade do produto.
\begin{itemize}
\item O produto não deverá ser ofensivo para os grupos étnicos e religiosos do pais de origem.
\item O produto deverá ser capaz de distinguir diferentes línguas e também diferentes sistemas de numeração, consoante o país de origem.
\item O produto deverá registar os feriados públicos do país.
\item O produto deverá se adaptar ao ambiente cultural do país, evitando símbolos, palavras e cores ofensivas para a sua cultura.
\end{itemize}

\subsection{Requisitos políticos}
Esta secção contém requisitos específicos relacionados com factores políticos que afectam a aceitabilidade do produto.
\begin{itemize}
\item O produto deverá fornecer todas as funcionalidades ao administrador da empresa (CEO).
\end{itemize}

\section{Requisitos Legais}
\subsection{Requisitos de conformidade}

Esta secção refere os requisitos legais do produto.
\begin{itemize}
\item O produto deverá ir de encontro à legislação em vigor no país.
\item O produto deverá respeitar a propriedade intelectual bem como os direitos de autor.
\item O produto deverá armazenar os dados seguirão as normas e directrizes dadas pelo sistema nacional de protecção de dados.
\end{itemize}

