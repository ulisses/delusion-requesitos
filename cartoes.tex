\pagebreak
\section{Requisitos Funcionais}

% Início do requisito 


\begin{minipage}{0.55\textwidth}
\begin{flushleft}\textbf{Requirement \#: 1}\end{flushleft}
\end{minipage}
\begin{minipage}{0.4\textwidth}
\end{minipage}

\begin{description}
\item \textbf{Description}:

O Qemu deve poder identificar chamadas do sistema e seu contexto, de forma automática. \\

\item \textbf{Rationale}:

Aqui descrição e justificação. \\
\item \textbf{Originator}:

Aqui o gajo que originou o requisito\\

\item \textbf{Fit Criterion}:

Aqui o critério de aceitação \\

\item \textbf{Priority}:

Must Have. \\

\end{description}

\pagebreak


% Fim do requisito 









% Início do requisito 


\begin{minipage}{0.55\textwidth}
\begin{flushleft}\textbf{Requirement \#: 2}\end{flushleft}
\end{minipage}
\begin{minipage}{0.4\textwidth}
\end{minipage}

\begin{description}
\item \textbf{Description}:

O Qemu deve poder filmar as acções do utilizador malicioso, automaticamente. \\


\item \textbf{Rationale}:

Aqui descrição e justificação. \\
\item \textbf{Originator}:

Aqui o gajo que originou o requisito\\

\item \textbf{Fit Criterion}:

Aqui o critério de aceitação \\

\item \textbf{Priority}:

Must Have. \\

\end{description}

\pagebreak


% Fim do requisito 


% Início do requisito 


\begin{minipage}{0.55\textwidth}
\begin{flushleft}\textbf{Requirement \#: 3}\end{flushleft}
\end{minipage}
\begin{minipage}{0.4\textwidth}
\end{minipage}

\begin{description}
\item \textbf{Description}:

O Qemu pode copiar qualquer ficheiro, que tenha sido interagido pelo utilizador malicioso, para o sistema onde está em funcionamento, 
ou, para um sistema remoto.

\item \textbf{Rationale}:

Aqui descrição e justificação. \\
\item \textbf{Originator}:

Aqui o gajo que originou o requisito\\

\item \textbf{Fit Criterion}:

Aqui o critério de aceitação \\

\item \textbf{Priority}:

Must Have. \\

\end{description}

\pagebreak


% Fim do requisito 



% Início do requisito 


\begin{minipage}{0.55\textwidth}
\begin{flushleft}\textbf{Requirement \#: 4}\end{flushleft}
\end{minipage}
\begin{minipage}{0.4\textwidth}
\end{minipage}

\begin{description}
\item \textbf{Description}:

O daemon collector do Delusion (DCD), deve poder capturar qualquer tipo de trâfego que passe pela máquina virtual.

\item \textbf{Rationale}:

Aqui descrição e justificação. \\
\item \textbf{Originator}:

Aqui o gajo que originou o requisito\\

\item \textbf{Fit Criterion}:

Aqui o critério de aceitação \\

\item \textbf{Priority}:

Must Have. \\

\end{description}

\pagebreak


% Fim do requisito 

% Início do requisito 


\begin{minipage}{0.55\textwidth}
\begin{flushleft}\textbf{Requirement \#: 5}\end{flushleft}
\end{minipage}
\begin{minipage}{0.4\textwidth}
\end{minipage}

\begin{description}
\item \textbf{Description}:

O utilizador consegue ver alertas, gerados pelo DCD, num browser.

\item \textbf{Rationale}:

Aqui descrição e justificação. \\
\item \textbf{Originator}:

Aqui o gajo que originou o requisito\\

\item \textbf{Fit Criterion}:

Aqui o critério de aceitação \\

\item \textbf{Priority}:

Must Have. \\

\end{description}

\pagebreak


% Fim do requisito 

% Início do requisito 


\begin{minipage}{0.55\textwidth}
\begin{flushleft}\textbf{Requirement \#: 6}\end{flushleft}
\end{minipage}
\begin{minipage}{0.4\textwidth}
\end{minipage}

\begin{description}
\item \textbf{Description}:

O utilizador consegue ver o trâfego que passou na máquina, capturado pelo DCD, num browser.

\item \textbf{Rationale}:

Aqui descrição e justificação. \\
\item \textbf{Originator}:

Aqui o gajo que originou o requisito\\

\item \textbf{Fit Criterion}:

Aqui o critério de aceitação \\

\item \textbf{Priority}:

Must Have. \\

\end{description}

\pagebreak


% Fim do requisito 


% Início do requisito 


\begin{minipage}{0.55\textwidth}
\begin{flushleft}\textbf{Requirement \#: 7}\end{flushleft}
\end{minipage}
\begin{minipage}{0.4\textwidth}
\end{minipage}

\begin{description}
\item \textbf{Description}:

O utilizador configurar os parâmetros de rede do HoneyPot, através, através dum browser.

\item \textbf{Rationale}:

Aqui descrição e justificação. \\
\item \textbf{Originator}:

Aqui o gajo que originou o requisito\\

\item \textbf{Fit Criterion}:

Aqui o critério de aceitação \\

\item \textbf{Priority}:

Must Have. \\

\end{description}

\pagebreak


% Fim do requisito 


% Início do requisito 


\begin{minipage}{0.55\textwidth}
\begin{flushleft}\textbf{Requirement \#: 8}\end{flushleft}
\end{minipage}
\begin{minipage}{0.4\textwidth}
\end{minipage}

\begin{description}
\item \textbf{Description}:

O utlizador consegue filtrar a informação, presente no visualizador, através dum browser.

\item \textbf{Rationale}:

Aqui descrição e justificação. \\
\item \textbf{Originator}:

Aqui o gajo que originou o requisito\\

\item \textbf{Fit Criterion}:

Aqui o critério de aceitação \\

\item \textbf{Priority}:

Must Have. \\

\end{description}

\pagebreak


% Fim do requisito 


% Início do requisito 


\begin{minipage}{0.55\textwidth}
\begin{flushleft}\textbf{Requirement \#: 9}\end{flushleft}
\end{minipage}
\begin{minipage}{0.4\textwidth}
\end{minipage}

\begin{description}
\item \textbf{Description}:

O utlizador consegue, escolher o tipo de gráfico usado para representar o trâfego e alertas, através dum browser.

\item \textbf{Rationale}:

Aqui descrição e justificação. \\
\item \textbf{Originator}:

Aqui o gajo que originou o requisito\\

\item \textbf{Fit Criterion}:

Aqui o critério de aceitação \\

\item \textbf{Priority}:

Must Have. \\

\end{description}

\pagebreak


% Fim do requisito 


% Início do requisito 


\begin{minipage}{0.55\textwidth}
\begin{flushleft}\textbf{Requirement \#: 10}\end{flushleft}
\end{minipage}
\begin{minipage}{0.4\textwidth}
\end{minipage}

\begin{description}
\item \textbf{Description}:

O utlizador consegue, configurar uma ou mais instâncias do HoneyPot, através dum browser.

\item \textbf{Rationale}:

Aqui descrição e justificação. \\
\item \textbf{Originator}:

Aqui o gajo que originou o requisito\\

\item \textbf{Fit Criterion}:

Aqui o critério de aceitação \\

\item \textbf{Priority}:

Must Have. \\

\end{description}

\pagebreak


% Fim do requisito 


% Início do requisito 


\begin{minipage}{0.55\textwidth}
\begin{flushleft}\textbf{Requirement \#: 11}\end{flushleft}
\end{minipage}
\begin{minipage}{0.4\textwidth}
\end{minipage}

\begin{description}
\item \textbf{Description}:

O administrador consegue criar e remover utilizadores, através dum browser.


\item \textbf{Rationale}:

Aqui descrição e justificação. \\
\item \textbf{Originator}:

Aqui o gajo que originou o requisito\\

\item \textbf{Fit Criterion}:

Aqui o critério de aceitação \\

\item \textbf{Priority}:

Must Have. \\

\end{description}

\pagebreak


% Fim do requisito 


% Início do requisito 


\begin{minipage}{0.55\textwidth}
\begin{flushleft}\textbf{Requirement \#: 12}\end{flushleft}
\end{minipage}
\begin{minipage}{0.4\textwidth}
\end{minipage}

\begin{description}
\item \textbf{Description}:

O administrador gerir as prioridades de utilizadores, através dum browser.


\item \textbf{Rationale}:

Aqui descrição e justificação. \\
\item \textbf{Originator}:

Aqui o gajo que originou o requisito\\

\item \textbf{Fit Criterion}:

Aqui o critério de aceitação \\

\item \textbf{Priority}:

Must Have. \\

\end{description}

\pagebreak


% Fim do requisito 

