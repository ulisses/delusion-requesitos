\chapter{Informação do Projecto}
\section{Motivação do Projecto}
\subsection{Contexto}
Nos dias que correm, é fundamental para as seguradoras a disponibilização de simuladores de seguro aos seus clientes, o que constitui um excelente modo de publicitar as suas inovações no desenho da fórmula de cálculo do prémio.
A inexistência no mercado de um simulador que agregue múltiplos seguros constituiu uma oportunidade para o desenvolvimento deste projecto, com grande potencial de interesse por parte das seguradoras, visto oferecer-lhes a vantagem de publicitar os seus descontos multi-seguro.
\subsection{Objectivo do Projecto}
O objectivo do projecto é informatizar um sistema na área de seguros, isto é, desde a simulação à contratação e configuração dos seguros. Com esta informatização pretende-se que os utilizadores do sistema consigam realizar o processo com mais facilidade. Também será desenvolvido um simulador multi-produto que junte às simulações unitárias toda a área de atribuição de descontos justificada pela multiplicidade de seguros.

\section{\emph{Stakeholders}}

\subsection{Cliente}
Deloitte (Representante: João Lúcio Borbinha)

\subsection{Comprador}
Companhias de Seguros

\subsection{Outros \emph{Stakeholders}}
Professores (\emph{input} de conhecimento técnico e supervisão do projecto):
\begin{itemize}
\item João Miguel Fernandes
\item Fernando Mário Martins
\item José Alberto Saraiva
\item Olga Maria Pacheco
\end{itemize}
\subsection{Os Utilizadores do Produto}
\begin{description}
\item \textbf{Utilizador 1}
\item Categoria: 

Recém-proprietários de um veículo automóvel
\item Responsabilidades do utilizador: 

Tem a obrigação de criar um seguro após a compra do carro.
\item Nível de conhecimento tecnológico: 

Conhecimentos básicos
\item Outras características importantes:

Idade igual ou superior a 18\\\\
\end{description}
\begin{description}
\item \textbf{Utilizador 2}
\item Categoria: 

Utilizadores descontentes com o seu seguro actual
\item Responsabilidades do utilizador: 

Tem um seguro já criado
\item Nível de conhecimento tecnológico: 

Variado
\item Outras características importantes: 

Vontade de mudança, aberto a sugestões. Idade igual ou superior a 18.\\\\
\end{description}

\subsection{\emph{Personas}}
\begin{description}
\item \textbf{Rui}

51 anos, casado, 2 filhos

Advogado, viaja com frequência

Gosta de futebol e de dançar

A família é o mais importante para ele

\item \textbf{Paula}

25 anos, solteira

Agente de seguros

Organizada e gosta de fazer compras

Gosta de fotografia e viajar

Gosta de novas tecnologias

\end{description}
\pagebreak
\subsection{Prioridade dos Intervenientes}

\begin{figure}[!htb]
     \centering
%     \includegraphics[scale=0.8]{images/piramide}
     \caption{Prioritização dos intervenientes}
\end{figure}

\subsection{Contribuição dos Intervenientes}
O cliente João Borbinha transmitiu ao grupo de trabalho conhecimento sobre o modelo de negócio em causa, e acompanhamento/aconselhamento periódico em relação às dúvidas resultantes do levantamento de requisitos. 

Pelo contacto realizado com um agente de seguros, obtivemos informação relevante sobre o processo da contratação nos seguros Automóvel e Saúde.

\subsection{Suporte e Manutenção}
Para suporte e manutenção estarão destacados os Administradores do sistema, técnicos que se espera terem conhecimento informático médio/alto, suficiente para executar a função com recurso apenas à documentação disponível (independentes dos developers originais).

