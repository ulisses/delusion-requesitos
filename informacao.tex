\chapter{Informação do Projecto}
\section{Motivação do Projecto}
\subsection{Contexto}
Esta ideia tem como principal sector de actividade a industria, ou seja irá vendido como produto. A ideia não
existe e por si só é o aglomerar de várias ideias, com alguns componentes que irão ser únicos e nunca antes feitos.
O produto final não existe. Esta ideia, graças a alguma experiência do promotor na área encontra-se numa fase de
projecto com alguns componentes já muito bem estudados e prototipados.
O actual mercado de negócio da ideia são todas as empresas de médio porte que prestem serviços na internet e que
tenham interesse em defender a sua rede informática interna onde os seus serviços habitam.
Os futuros mercados de negócio a 5 anos são essencialmente os mercados internacionais, Europa, Ásia e America
do Norte essencialmente, onde existam os mesmos problemas supracitados.
Apresentação da Ideia
Esta ideia está relacionada com o tema de segurança e tem como principal objectivo criar um produto (chamemos
caixa, um PC comum) que habita na ligação de uma rede interna à internet. Esta caixa irá ter um Linux, com a
capacidade de garantir a segurança da rede interna. Para que o grau de segurança seja aumentado este produto irá
ter os já comuns sistemas de protecção largamente utilizados, como firewall, IPS (Intrusion prevention system) e
vários sistemas de log ao nível de rede. Relativamente a este ponto, irão ser criadas interfaces web para configurar
e monitorizar toda a actividade maliciosa que a rede interna sofre.
Para além disso irá ser desenvolvido um sistema honeypot que basicamente é um sistema que cria falsos serviços
na rede com várias vulnerabilidades conhecidas activas. Assim, conseguiremos atrair um atacante mais
rápidamente para esses serviços (e monitorizar a sua actividade), deixando assim os serviços reais livres de
interesse por parte dos atacantes. Digamos que com isto queremos não só defender como também mostrar que
temos alguns problemas para distrair a atenção de uma possível entidade maliciosa que poderia inflingir ataques na
rede interna.
Em nenhum momento estes serviços falsos irão criar problemas com os serviços reais, visto serem coisas
independentes.
Portanto, temos assim dois componentes que necessitam de ser investigados e desenvolvidos.
Por um lado temos todas as camadas de segurança, que irão ser estudadas e integradas no nosso produto, com a
acrescente de criar uma interface fácil e intuitiva para um Eng. de redes utilizar, configurar e fazer inspecção.
Por outro lado temos a construção de um honeypot com a capacidade de criar serviços falsos na rede, atrair
atacantes e ainda loggar toda a actividade maliciosa que for feita contra si. Este componente consiste numa
máquina virtual (dentro do sistema operativo real da caixa) que é colocada numa DMZ da rede (para poder ser
vista na internet) e tem de ter a capacidade de ser atacada e guardar o que foi feito, para conseguir criar um perfil
do ataque e entender que tipos de ataques estão “na moda”.
Esta parte de entender os ataques mais recentes utilizados tem como principal objectivo ajudar ao melhoramento
do produto através da criação de novas regras tanto na firewall como no IPS.
Este último ponto é fundamental: Os tipos de ataques estão sempre a mudar, as regras de uma firewall construída
há 5 anos já não são válidas para os dias de hoje. Ao conseguir traçar um perfil do ataque que o honeypot sofreu,
conseguimos garantir que estamos sempre actualizados sobre de que maneira conseguimos melhorar o nosso
produto.
Embora isto não interesse imediatamente ao cliente, interessa a quem produz as caixas pois assim conseguirá
melhorar o produto garantindo a sua viabilidade/utilidade no futuro.
Pontos de vista técnicos da ideia
A componente de segurança da ideia passará essencialmente por investigação e aprendizagem da ferramentas mais
comuns actualmente para protecção em redes, sejam iptables para firewalls ou o snort para IPS (existem outros que
precisam de ser vistos e estudados).
A inovação nesta ideia referente a esta componente está na criação de um interface web para a gestão destas
camadas de segurança.
Por outro lado temos a criação de um honeypot afim de forjar serviços na rede, que sejam visiveis para a internet.
É interessante dizer que existem duas familias de honeypots os de AI (alta-interactividade) e os de BI (baixainteractividade).
O que os distingue é a capacidade de forjar melhor ou pior os serviços vulneráveis na rede.
Os honeypots de BI são serviços que são forjados através de simples scripts perl/bash/python e que por serem
scripts teem limitações naturais, pois uma vez um atacante estando a interagir com um serviço ficticio telnet por
exemplo, ele espera que todos os comandos telnet estejam disponiveis para ele, o que dificulta a construção do
script, pois que o irá fazer terá que ter em atenção tudo o que um serviço telnet pode fazer. No entanto estes tipos
de honeypot são interessantes para perceber levemente a natureza do ataque.
Por outro lado os honeypots de AI costumam ser sistemas operativos reais, com falhas reais. O grande problema
que nasce é: como inpeccionar o que se passa dentro de um sistema operativo sem que nenhum utilizador (até
mesmo root) consiga perceber.
O objectivo desta ideia aqui descrita é criar este honeypot AI e existem duas maneiras de o fazer:
Através de uma máquina virtual Vmware por exemplo, onde deixamos o atacante fazer o que quiser dentro dessa
máquina virtual e inpeccionamos todas as chamadas que são feitas ao VMware. O que pode ser muito dificil.
Ou entao a adulteração do sistema operativo para que consigamos inspeccionar o que o atacante está a fazer e ao
mesmo tempo fazê-lo sem que este se aperceba de que está a ser observado. Isto é mais fácil, no entanto passa por
adulteração de várias system calls so Linux entre outras.

Nos dias que correm, é fundamental para as seguradoras a disponibilização de simuladores de seguro aos seus clientes, o que constitui um excelente modo de publicitar as suas inovações no desenho da fórmula de cálculo do prémio.
A inexistência no mercado de um simulador que agregue múltiplos seguros constituiu uma oportunidade para o desenvolvimento deste projecto, com grande potencial de interesse por parte das seguradoras, visto oferecer-lhes a vantagem de publicitar os seus descontos multi-seguro.
\subsection{Objectivo do Projecto}
O objectivo do projecto é informatizar um sistema na área de seguros, isto é, desde a simulação à contratação e configuração dos seguros. Com esta informatização pretende-se que os utilizadores do sistema consigam realizar o processo com mais facilidade. Também será desenvolvido um simulador multi-produto que junte às simulações unitárias toda a área de atribuição de descontos justificada pela multiplicidade de seguros.

\section{\emph{Stakeholders}}

\subsection{Cliente}
Vision Space Technologies (Representante: Ulisses Araújo Costa)

\subsection{Comprador}
Empresas que tenham uma rede informática de tamanho médio e que procurem soluções de prevenção e compreensão de ataques a nivel de segurança de redes.
Portanto este produto é dirigido a empresas que podem ser ou não de informática, mas que tenham staff técnico para gerir a rede e os serviços
informáticos que a empresa tem.

\subsection{Outros \emph{Stakeholders}}
Professores (supervisão do projecto):
\begin{itemize}
\item João Miguel Fernandes
\item Victor Francisco Fonte
\item António Nestor Ribeiro
\item Vitor Manuel Rodrigues Alves
\end{itemize}
\subsection{Os Utilizadores do Produto}
\begin{description}
\item \textbf{Utilizador 1}
\item Categoria: 

Administrador de Redes informáticas
\item Responsabilidades do utilizador: 

Tem a obrigação de ajudar a melhorar a segurança da rede que administra.
\item Nível de conhecimento tecnológico: 

Conhecimentos avançados a nível de rede, ferramentas de inspecção de rede e administração Linux.
\end{description}

%\subsection{Prioridade dos Intervenientes}
%\begin{figure}[!htb]
%     \centering
%     \includegraphics[scale=0.8]{images/piramide}
%     \caption{Prioritização dos intervenientes}
%\end{figure}

\subsection{Contribuição dos Intervenientes}
A empresa XPTO lda. tem uma rede que liga os seus 40 funcionários a serviços internos e externos à empresa. Tem também um administrador de rede que
quer aumentar o grau de percepção sobre os ataques que esta rede sofre afim de tomar medidas para aumentar a segurança, sem por em
causa o bom funcionamento dos serviços informáticos.\\
A XPTO lda. compra o Delusion e instala na sua rede.

\subsection{Suporte e Manutenção}
Para suporte e manutenção estarão destacados Administradores de sistemas, técnicos que se espera terem conhecimento informático médio/alto,
suficiente para executar a função com recurso apenas à documentação disponível (independentes dos developers originais).

