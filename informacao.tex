\chapter{Informação do Projecto}
\section{Motivação do Projecto}
\subsection{Contexto}
Este projecto dirige-se a todas as empresas de tamanho médio que teêm serviços na internet e na sua rede interna e que
tenham interesse em defender a sua rede informática interna onde os seus serviços habitam.
Portanto o produto que vai nascer deste projecto destina-se à area de Segurança de redes informáticas.\\
O produto final terá a forma de um appliance a ser instalado na rede destino a assegurar.

%Nos dias que correm, é fundamental para as seguradoras a disponibilização de simuladores de seguro aos seus clientes, o que constitui um excelente
%modo de publicitar as suas inovações no desenho da fórmula de cálculo do prémio.
%A inexistência no mercado de um simulador que agregue múltiplos seguros constituiu uma oportunidade para o desenvolvimento deste projecto,
%com grande potencial de interesse por parte das seguradoras, visto oferecer-lhes a vantagem de publicitar os seus descontos multi-seguro.

\subsection{Objectivo do Projecto}
Esta ideia tem como principal objectivo criar um produto sob a forma de appliance
que habita numa zona da rede interna, preferencialmente mais próximo da ligação à internet.\\
O produto terá como objectivo colectar a informação de rede e também de uma máquina simulada especifica, onde serão instalados serviços com falhas conhecidas.
Esta máquina é um honeypot, com um sistema operativo real e serviços reais com falhas reais, com a particularidade de estar a ser inspeccionada
pela appliance e consequentemente gravar toda a actividade que um atacante faça dentro dela.\\
Iremos ainda criar uma outra máquina com a capacidade de inspeccionar todo o tráfego malicioso que atravessa a rede da empresa e com isto informar o
adminsitrador de rede sobre o uso indevido que a rede sofre. Estes inspectores estarão espalhados pelos vários segmentos da rede interna afim de 
oferecerem uma maior cobertura de análise ao nível da actividade de rede.\\
Assim, quando um atacante entrar na rede iremos conseguir gravar e mostrar de forma clara a informação sobre o ataque que a empresa sofreu desde a actividade
que aconteceu na rede até ao que o atacante fez dentro da máquina honeypot.\\

Quando se fala em máquinas, falamos essencialmente de máquinas virtualizadas numa máquina física - appliance de rede,
uma vez que interessa baixar o custo final para o cliente.\\

O objectivo é transformar actividade complexa dentro da rede e dentro do honeypot (uma máquina que está a ser constantemente inspeccionada) em
informação útil que permita a um administrador de redes de uma empresa perceber os ataques que houve e tomar decições em tempo útil sobre como proteger a rede.

%\subsection{Objectivo do Projecto}
%O objectivo do projecto é informatizar um sistema na área de seguros, isto é, desde a simulação à contratação e configuração dos seguros.
%Com esta informatização pretende-se que os utilizadores do sistema consigam realizar o processo com mais facilidade.
%Também será desenvolvido um simulador multi-produto que junte às simulações unitárias toda a área de atribuição de descontos justificada pela multiplicidade de seguros.

\section{\emph{Stakeholders}}

\subsection{Cliente}
Vision Space Technologies (Representante: Ulisses Araújo Costa)

\subsection{Comprador}
Empresas que tenham uma rede informática de tamanho médio e que procurem soluções de prevenção e compreensão de ataques a nivel de segurança de redes.
Portanto este produto é dirigido a empresas que podem ser ou não de informática, mas que tenham staff técnico para gerir a rede e os serviços
informáticos que a empresa tem.

\subsection{Outros \emph{Stakeholders}}
Professores (supervisão do projecto):
\begin{itemize}
\item João Miguel Fernandes
\item Victor Francisco Fonte
\item António Nestor Ribeiro
\item Vitor Manuel Rodrigues Alves
\end{itemize}
\subsection{Os Utilizadores do Produto}
\begin{description}
\item \textbf{Utilizador 1}
\item Categoria: 

Administrador de Redes informáticas
\item Responsabilidades do utilizador: 

Tem a obrigação de ajudar a melhorar a segurança da rede que administra.
\item Nível de conhecimento tecnológico: 

Conhecimentos avançados a nível de rede, ferramentas de inspecção de rede e administração Linux.
\end{description}

%\subsection{Prioridade dos Intervenientes}
%\begin{figure}[!htb]
%     \centering
%     \includegraphics[scale=0.8]{images/piramide}
%     \caption{Prioritização dos intervenientes}
%\end{figure}

\subsection{Contribuição dos Intervenientes}
A empresa XPTO lda. tem uma rede que liga os seus 40 funcionários a serviços internos e externos à empresa. Tem também um administrador de rede que
quer aumentar o grau de percepção sobre os ataques que esta rede sofre afim de tomar medidas para aumentar a segurança, sem por em
causa o bom funcionamento dos serviços informáticos.\\
A XPTO lda. compra o Delusion e instala na sua rede.

\subsection{Suporte e Manutenção}
Para suporte e manutenção estarão destacados Administradores de sistemas, técnicos que se espera terem conhecimento informático médio/alto,
suficiente para executar a função com recurso apenas à documentação disponível (independentes dos developers originais).

