\chapter{Requisitos Não-Funcionais}
\section{Requisitos de Aspecto e Percepção}
\subsection{Requisitos de Estilo}
\begin{description}
\item[Requisito] O produto deve ter uma aparência profissional.
\item[Fit Criterion] Opinião favorável de todos os criadores do produto.
\end{description}

\section{Requisitos Humanos e Usabilidade}
\subsection{Requisitos de Usabilidade}
\begin{description}
\item[Requisito] O produto não deve ser aborrecido ou maçador de usar.
\item[Fit Criterion] Não abusar no uso de cores muito brilahntes, tamanho de letra sempre igual.
\end{description}

\section{Requisitos de Aprendizagem}
\begin{description}
\item[Requisito] Um administrador de redes familiarizado com Linux e ficheiros log deve conseguir utilizar o produto facilmente.
\item[Fit Criterion] Um grupo de pessoas que já fizeram administração de redes devem conseguir ambientar-se ao front-end no espaço de uma hora.
\end{description}

\section{Requisitos de Performance}
\subsection{Requisitos de Velocidade e Latência}
\begin{itemize}
\item A consulta da informação guardada não deverá demorar mais que 10 segundos.
\item A aplicação de filtros até encontrar a informação que se pretende não deverá demorar mais que 10 minutos.
\end{itemize}

\subsection{Requisitos de Precisão}
\begin{itemize}
\item O relógio do sistema trabalhará em GMT.
\end{itemize}

\subsection{Requisitos de Fiabilidade e Disponibilidade}
\begin{itemize}
\item O produto deverá funcionar sem falhas inesperadas em 99 de 100 utilizações.
\end{itemize}

\subsection{Requisitos de Longevidade}
\begin{itemize}
\item É esperado que o produto desempenhe a sua função correctamente de forma constante, até que a regulamentação da área de domínio que rege o seu funcionamento seja alterada.
\end{itemize}

\section{Requisitos Operacionais e Ambientais}
\subsection{Ambiente Físico Esperado}
\begin{itemize}
\item É esperado que este produto seja instalado numa rede num ambiente de uma empresa, dentro de um espaço fechado com ar condicionado e sistemas contra falhas de energia.
\end{itemize}

\subsection{Requisitos de Interacção com Sistemas Adjacentes}
\begin{description}
\item[Requisito] Este produto deverá ser utilizável nos quatro \emph{browsers} mais populares.
\item[Fit Criterion] O produto deveria ser utilizado com sucesso em acessos via Internet Explorer, Firefox, Google Chrome e Safari actualmente.
É expectável que estes continuem os quatro \emph{browsers} mais populares aquando da data de lançamento dos protótipos.
\end{description} 

\section{Requisitos de Manutenção e Suporte}
\subsection{Requisitos de Manutenção}
\begin{description}
\item[Requisito] O produto deve poder ser mantido por utilizadores e \emph{developers} que não os originais.
\item[Fit Criterion] O produto deve ser auto-suficiente em termos de suporte para utilizadores e \emph{developers} alheios ao desenvolvimento do produto, com recurso apenas ao manual de utilização.
\end{description}

