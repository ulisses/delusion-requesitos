\chapter{Requisitos Não-Funcionais}
\minitoc

\section{Requisitos de Aspecto e Percepção}

% Início de requisito

\subsection{Requisitos de Estilo}
\begin{description}
\item[Requisito] O visualizador presente no produto deve ter uma aparência profissional.
\item[Motivação] O utilizador irá assim obter uma experiência de utilização mais positiva e irá sentir maior confiança e robustez
ao usar o produto.
\item[Fit Criterion] Opinião favorável de todos os stakeholders.
\end{description}


% Fim de requisito



% Início de requisito

\section{Requisitos Humanos e Usabilidade}
\subsection{Requisitos de Usabilidade}
\begin{description}
\item[Requisito] O visualizador do produto deverá ser dinamico e não mostrar a informação sempre da mesma forma, sendo capaz de destacar informação mais importante de informação menos importante.
\item[Motivação] Este requisito é importante para que o utilizador não se sinta aborrecido a usar o sistema, bem como, 
para que a informação mais importante não lhe passe facilmente despercebida.  
\item[Fit Criterion] Opinião favorável dos stakeholders.
\end{description}


% Fim de requisito


% Início de requisito


\section{Requisitos de Aprendizagem}
\begin{description}
\item[Requisito] Um administrador de redes familiarizado com Linux e ficheiros log deve conseguir utilizar o produto facilmente.
\item[Motivação] Isto é importante para que não haja grandes custos a treinar uma pessoa para usar o sistema, bem como, para que esta consiga usar 
todas as potencialidades que o produto disponibiliza.
\item[Fit Criterion] Um grupo de pessoas que já fizeram administração de redes, devem conseguir utilizar em pleno o sistema no espaço de uma hora.
\end{description}


% Fim de requisito




% Início de requisito

% Este eu não sei que queres dizer com ele por isso vou deixar assim 
\section{Requisitos de Performance}
\subsection{Requisitos de Velocidade e Latência}
\begin{itemize}
\item A consulta da informação guardada não deverá demorar mais que 10 segundos.
\item A aplicação de filtros até encontrar a informação que se pretende não deverá demorar mais que 10 minutos.
\end{itemize}

% Fim de requisito




% Início de requisito
% Este nao acho que faça muito sentido, um japonês tem que utilizar o horário GMT ?
\subsection{Requisitos de Precisão}
\begin{itemize}
\item O relógio do sistema trabalhará em GMT.
\end{itemize}

% Fim de requisito



% Início de requisito
\subsection{Requisitos de Fiabilidade e Disponibilidade}
\begin{itemize}
\item O produto deverá funcionar sem falhas inesperadas em 99 de 100 utilizações.
\end{itemize}

% Fim de requisito



% Início de requisito

\begin{description}
\item[Requisito] O produto não pode deixar de registar eventos maliciosos.
\item[Motivação] Caso o produto não consiga registar todos os eventos maliciosos vai ser pouco fiável e pouco útil, pois, a principal utilidade
do sistema é que este seja capaz de registar todos os eventos maliciosos.
\item[Fit Criterion] Em 10000 eventos ocorridos, o sistema tem q ser registar pelo menos 9999. 
\end{description}

% Fim de requisito

% Início de requisito

\begin{description}
\item[Requisito] O visualizador do sistema tem, sempre, que estar disponível para o utilizador.
\item[Motivação] Caso o visualizador não esteja disponível o sistema perde praticamente toda a sua utilidade, já que, sem este fica muito mais 
complicado ver a informação coletada.
\item[Fit Criterion] Salvo falhas de energia, hardware, terceiros e catástrofes naturais o visualizador deve estar sempre disponível ao utilizador. 
\end{description}

% Fim de requisito


% Acho que os dois seguintes nao interessam ate aos comments
\subsection{Requisitos de Longevidade}
\begin{itemize}
\item É esperado que o produto desempenhe a sua função correctamente de forma constante, até que a regulamentação da área de domínio que rege o seu funcionamento seja alterada.
\end{itemize}

\section{Requisitos Operacionais e Ambientais}
\subsection{Ambiente Físico Esperado}
\begin{itemize}
\item É esperado que este produto seja instalado numa rede num ambiente de uma empresa, dentro de um espaço fechado com ar condicionado e sistemas contra falhas de energia.
\end{itemize}
%%%%%%%%%%



% Início de requisito

\begin{description}
\item[Requisito] O visualizador tem que poder ser utilizado nos 4 browsers mais importantes.
\item[Motivação] Se restringirmos os browsers capazes de suportar o visualizador, podemos também estar a restringir 
o conjunto de utilizadores dispostos a usar o visualizador.
\item[Fit Criterion] O visualizador tem que poder ser usado, indiferentemente, no Internet Explorer 8 e posteriores, no Mozilla Firefox 3 e 
posteriores, no Google Chrome 10 e posteriores e finalmente Apple Safari 4 e posteriores.
\end{description}

% Fim de requisito

\section{Requisitos de Manutenção e Suporte}
\subsection{Requisitos de Manutenção}
\begin{description}
\item[Requisito] O produto deve poder ser mantido por utilizadores e \emph{developers} que não os originais.
\item[Fit Criterion] O produto deve ser auto-suficiente em termos de suporte para utilizadores e \emph{developers} alheios ao desenvolvimento 
do produto, com recurso apenas ao manual de utilização.
\end{description}

