\chapter{Requisitos Não-Funcionais}
\section{Requisitos de Aspecto e Percepção}
\subsection{Requisitos de Estilo}
\begin{description}
\item \textbf{Requisito}: O produto deve ter uma aparência profissional.\\

\item \textbf{Fit Criterion}: Opinião favorável do cliente João Borbinha.
\end{description}



\section{Requisitos Humanos e Usabilidade}
\subsection{Requisitos de Usabilidade}
\begin{description}
\item \textbf{Requisito}: O produto deve ser facilmente utilizado por adultos de idade avançada.\\

\item \textbf{Fit Criterion}: Tamanho de letra sempre igual ou superior a 12, 95\% conseguirem completar uma simulação.\\\\\\

\item \textbf{Requisito}: O produto deve prevenir a inserção de dados inadequados/mal formatados por parte do utilizador.\\

\item \textbf{Fit Criterion}: Taxa de erros abaixo de 5\% após 1mês de utilização.
\end{description}

\subsection{Requisitos de Aprendizagem}
\section{Learning Requirements}
\begin{description}
\item \textbf{Requisito}: Um adulto familiarizado com a Internet deve conseguir utilizar o produto facilmente.\\

\item \textbf{Fit Criterion}: Um grupo de utilizadores regulares da Internet com os dados necessários para simulação deve conseguir efectuar registo e simulações no espaço de uma hora.\\\\\\

\item \textbf{Requisito}: As empresas de seguros devem poder configurar os seus produtos de forma independente.\\

\item \textbf{Fit Criterion}: Um colaborador de empresa de seguros deve poder adicionar produtos, coberturas e descontos da empresa apenas com recurso ao manual de utilização.
\end{description}


\subsection{Requisitos de Compreensão e Comportamento}
\begin{description}
\item \textbf{Requisito}: Linguagem fácil de compreender por adultos iletrados.\\

\item \textbf{Fit Criterion}: Linguagem informal; Termos do glossário da área de seguros sempre acompanhados de breve definição quando o cursor do rato é colocado sobre os mesmos.
\end{description}


\subsection{Requisitos de Acessibilidade}
\begin{description}
\item \textbf{Requisito}: O produto deve poder ser usado por utilizadores com dificuldades visuais.\\

\item \textbf{Fit Criterion}: Opção para redimensionamento do tipo de letra (\textsc{a}A)
\end{description}


\section{Requisitos de Performance}
\subsection{Requisitos de Velocidade e Latência}
\begin{itemize}
\item A consulta de simulação guardada não deverá demorar mais que 10 segundos.

\item A apresentação da lista de marcas/modelos/especificação automóvel durante o processo de simulação não deverá demorar mais que 5 segundos cada.

\item O registo e processamento de uma simulação não deverá demorar mais que 10 segundos.
\end{itemize}


\subsection{Requisitos de Precisão}
\begin{itemize}
\item Todos os valores monetários apresentados devem possuir precisão mínima de duas casas decimais, quando a moeda o permitir.

\item O produto utilizará o sistema métrico.

\item O relógio do sistema trabalhará em GMT.
\end{itemize}


\subsection{Requisitos de Fiabilidade e Disponibilidade}
\begin{itemize}
\item O produto deverá funcionar sem falhas inesperadas em 99 de 100 utilizações.
\end{itemize}

\subsection{Requisitos de Robustez}
\begin{itemize}
\item O produto deverá fazer backups diários (5h00) da informação guardada em Bases de Dados.
\end{itemize}


\subsection{Requisitos de Longevidade}
\begin{itemize}
\item É esperado que o produto desempenhe a sua função correctamente de forma constante, até que a regulamentação da área de domínio que rege o seu funcionamento seja alterada.
\end{itemize}


\section{Requisitos Operacionais e Ambientais}
\subsection{Ambiente Físico Esperado}
\begin{itemize}
\item É esperado que este produto seja instalado em computadores num ambiente típico de casa/escritório, dentro de portas e com condições climatéricas moderadas.
\end{itemize}


\subsection{Requisitos de Interacção com Sistemas Adjacentes}
\begin{description}
\item \textbf{Requisito}: Este produto deverá ser utilizável nos quatro \emph{browsers} mais populares.\\

\item \textbf{Fit Criterion}: O produto deveria ser utilizado com sucesso em acessos via Internet Explorer, Firefox, Google Chrome e Safari actualmente. É expectável que estes continuem os quatro \emph{browsers} mais populares aquando da data de lançamento dos protótipos.
\end{description} 

\section{Requisitos de Manutenção e Suporte}
\subsection{Requisitos de Manutenção}
\begin{description}
\item \textbf{Requisito}: O produto deve poder ser mantido por utilizadores e \emph{developers} que não os originais.\\

\item \textbf{Fit Criterion}: O produto deve ser auto-suficiente em termos de suporte para utilizadores e \emph{developers} alheios ao desenvolvimento do produto, com recurso apenas ao manual de utilização.
\end{description}

\subsection{Requisitos de Adaptação}
\begin{itemize}
\item O produto deve poder ser utilizado em sistemas Windows, Linux, OS X.
\end{itemize}
