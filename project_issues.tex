\chapter{Questões do Projecto}
\minitoc
\section{Pontos em aberto}
Relativamente a este projecto existem algumas questões importantes que necessitam de um esclarecimento prévio por parte do cliente.
A resposta a estas questões é essencial para o planeamento e para uma melhor especificação do sistema em si:

\subsection{Especificação entidades sistema}
Da análise prévia já efectuada, foram identificados intervenientes neste sistema:

\begin{itemize} 
\item Utilizador (anónimo/registado)
\item Administrador (empresa de seguros)
\item Mediador (Corretor/Agente/Mediador de seguros ligado)
\end{itemize}

É necessário especificar se existe mais alguma entidade a ser considerada que tenha interacção no sistema a desenvolver e quais as suas competências. Relativamente à entidade mediador esclarecer a divisão feita: Mediador em Corretor, Agente, e Mediador de Seguros Ligado (actualização do antigo estatuto de Angariador no Decreto-Lei n 144/2006, de 31 de Julho).

\subsection{Sistema por empresa}
Clarificar se a empresa de seguros necessita de implementar um sistema único para várias possíveis sucursais ou se a configuração é feita unitariamente sempre para cada empresa/sucursal.


\section{Novos Problemas}

\subsection{Incorporação aplicações}
Para aplicações actualmente existentes e sobre os quais terá que haver uma partilha ou transferência de dados terá também que ser efectuada uma análise, nomeadamente:

\begin{itemize} 
\item Identificação e levantamento aplicações sobre as quais haverá partilha ou fluxo de dados
\item Identificação dos dados a serem partilhados para cada aplicação
\item Definição de métodos e formatos de partilha de dados para cada aplicação
\item Definição de tarefas e calendarização  
\end{itemize} 

\subsection{Manutenção do sistema}
O seguimento do sistema, depois da sua instalação é assegurado por um serviço de suporte ao produto, disponibilizado pela empresa fornecedora do sistema. Contudo, esse suporte não cobre situações externas que possam influenciar o funcionamento deste, tais como problemas com infra-estrutura e comunicações, que terão que ser asseguradas pela empresa tomadora do sistema. 

\section{Tarefas}
\subsection{Planeamento}
Estão previstas as fases abaixo descritas, respectivas tarefas e resultados:
\begin{sidewaystable}[!p]
\begin{center}
\setlength{\tabcolsep}{5pt}
\begin{tabular}{|p{1cm}|p{1.2cm}|p{1.2cm}|p{1.2cm}|p{1.2cm}|p{1.2cm}|p{1.2cm}|p{1.2cm}|p{1.2cm}|p{2.4cm}|p{2.4cm}|}
\hline & \multicolumn{8}{|c|}{\T \B \textbf{Fases Iniciais}} & \multicolumn{1}{|c|}{\textbf{Formação}} & \multicolumn{1}{|c|}{\textbf{Suporte}}\\
\cline{2-9} & \multicolumn{2}{|c|}{\T \B \textbf{Concepção}} & \multicolumn{2}{|c|}{\textbf{Elaboração}} & \multicolumn{2}{|c|}{\textbf{Construção}} & \multicolumn{2}{|c|}{\textbf{Transição}} & \multicolumn{1}{|c|}{\textbf{ }} & \multicolumn{1}{|c|}{\textbf{ }}\\
\hline \multirow{6}{3cm}{\begin{sideways}\T \B \parbox{3cm}{Tarefas (Peso)}\end{sideways}} & MN & (+) & MN & (+) & MN & (o) & MN & (-) & & \\
\cline{2-9} & AR & (+) & AR & (+) & AR & (o) & AR & (-) & & \\
\cline{2-9} & ED & (*) & ED & (*) & ED & (+) & ED & (o) & Formação & Suporte \\
\cline{2-9} & IMP & (o) & IMP & (*) & IMP & (+) & IMP & (o) & (+) & (+)\\
\cline{2-9} & TST & (-) & TST & (o) & TST & (*) & TST & (+) & & \\
\cline{2-9} & INST & (-) & INST & (-) & INST & (-) & INST & (+) & & \\
\hline \multirow{5}{3cm}{\begin{sideways}\T \B \parbox{3cm}{Resultados}\end{sideways}} & \multicolumn{2}{|p{2cm}|}{Domínio do Sistema} & \multicolumn{2}{|p{2cm}|}{Descrição e Arquitectura do Sistema} & \multicolumn{2}{|p{2cm}|}{Protótipo do Sistema} & \multicolumn{2}{|p{2cm}|}{Migração, incorporação das aplicações existentes} & Formação dos Utilizadores & Suporte Aplicacional\\
\cline{2-11}\T \B & \multicolumn{2}{|p{2cm}|}{Identificação dos Stakeholders} & \multicolumn{2}{|p{2cm}|}{Descrição do negócio e riscos do projecto} & \multicolumn{2}{|p{2cm}|}{Validação} & \multicolumn{2}{|p{2cm}|}{Instalação} &  & Tarefas de manutenção do sistema\\
\cline{2-11}\T \B & \multicolumn{2}{|p{2cm}|}{Análise de Requisitos} & \multicolumn{2}{|p{2cm}|}{Plano de Desenvolvimento} & \multicolumn{2}{|p{2cm}|}{Testes} & \multicolumn{2}{|p{2cm}|}{Entrada em produção} & & Migrações\\
\cline{2-11}\T \B & \multicolumn{2}{|p{2cm}|}{Estimativa inicial de custos} & \multicolumn{2}{|p{2cm}|}{Especificação do sistema} & \multicolumn{2}{|p{2cm}|}{ } & \multicolumn{2}{|p{2cm}|}{ } & & \\
\cline{2-11}\T \B & \multicolumn{2}{|p{2cm}|}{Avaliação inicial de prioridades, riscos e processo de desenvolvimento} & \multicolumn{2}{|p{2cm}|}{Modelo da aplicação} & \multicolumn{2}{|p{2cm}|}{ } & \multicolumn{2}{|p{2cm}|}{ } & & \\
 \hline
\end{tabular}
 \caption{Tabela de Planeamento}
\end{center}
\end{sidewaystable}
\clearpage
\subsection{Prazos}
De seguida faz-se uma estimativa dos diferentes prazos para as fases acima expostas:

\begin{table}[!h]
\begin{center}
\setlength{\tabcolsep}{2pt}
\begin{tabular}{|p{1.5cm}|p{1.5cm}|p{1.5cm}|p{1.5cm}|p{1.5cm}|p{2.2cm}|p{2.2cm}|}
\hline & \multicolumn{4}{|c|}{\T \B \textbf{Fases Iniciais}} & \multicolumn{1}{|c|}{\textbf{Formação}} & \multicolumn{1}{|c|}{\textbf{Suporte}}\\
\cline{2-5} & \multicolumn{1}{|c|}{ \T \B \textbf{Concepção}} & \multicolumn{1}{|c|}{\textbf{Elaboração}} & \multicolumn{1}{|c|}{\textbf{Construção}} & \multicolumn{1}{|c|}{\textbf{Transição}} & \multicolumn{1}{|c|}{\textbf{ }} & \multicolumn{1}{|c|}{\textbf{ }}\\
\hline \T \B Data Limite & Janeiro 2010 & Fevereiro 2010 & Abril 2010 & Maio 2010 & A especificar & A especificar\\
 \hline
\end{tabular}
 \caption{Tabela de Prazos}
\end{center}
\end{table}

\section{Migração para o novo produto}
\subsection{Requisitos}
Para garantir a migração para o novo produto a empresa tomadora do produto terá que garantir algumas condições necessárias, nomeadamente:

\begin{itemize} 
\item Infra-estrutura: local adequado onde irá ser albergado o(s) servidor(es) que serviram de suporte ao sistema, com condições adequadas ao seu funcionamento.
\item Requisitos de hardware: fornecimento da máquina, ou máquinas onde serão instalados o servidor de base de dados e o servidor aplicacional.
\item Requisitos de software: fornecimento do software base de suporte aplicacional, onde se inclui por exemplo o sistema operativo, componentes necessários para o funcionamento do sistema e base de dados.
\end{itemize}

É necessário também incluir a nova aplicação no sistema de backups da empresa, quer a nível aplicacional como de dados.
Em caso de migração ou incorporação de aplicações existentes terá que haver uma análise caso a caso para análise dos requisitos necessários para efectivar essa mesma migração ou incorporação.
 
\section{Riscos}
Todos os projectos envolvem riscos. O risco não é necessariamente uma coisa má, pelo que, sem riscos não existe avanço nos projectos. No entanto, existem diferenças entre os riscos, ou seja, entre os riscos incontrolados e os riscos passíveis de ser geridos. Os riscos só são maus se forem ignorados e começarem a dar problemas.
A gestão de riscos implica a apreciação daqueles riscos que são mais prováveis de aplicar no projecto, decidindo o curso a tomar se esses riscos se tornarem problemas, e também na monitorização dos projectos de forma a detectar-se precocemente sinais sobre os riscos, de maneira a evitar que se tornem problemas.
Abaixo passamos a citar a lista dos riscos mais prováveis e mais sérios de aparecer no projecto. Cada risco incluiu uma probabilidade de ele se tornar um problema.

\begin{itemize}
\item \textbf{Mudança dos requisitos} 
Uma mudança nos requisitos acarreta um significativo risco no projecto, significando alterações na modelação dos requisitos iniciais. A probabilidade deste risco se tornar problema é alta, visto que não depende exclusivamente da equipa de desenvolvimento, e portanto, torna-se volátil a sua execução.

\item \textbf{Pressão excessiva do cronograma} 
O cronograma é um risco calculado que poderá se tornar problema quando se deixa de controlar o tempo. A probabilidade deste risco se tornar problema é relativamente baixa visto que todos os envolvidos no projecto estão cientes dos prazos a cumprir.

\item \textbf{Requisitos problemáticos do utilizador}
Por vezes os requisitos dos utilizadores tornam-se riscos para o projecto visto que muitos deles só são vistos como não exequíveis numa fase adiantada do projecto. A probabilidade de se tornar problema pode ser alta se os requisitos problemáticos não forem detectados inicialmente.

\item \textbf{Qualidade baixa}
O risco da qualidade do produto só será problema se o produto desenvolvido não corresponder ao expectável pelo cliente. No entanto, esse risco é controlável através dos diversos pontos de controlo do projecto ao longo do seu desenvolvimento. A probabilidade de se tornar um problema é baixa.
\end{itemize}

Tendo em conta a análise aos riscos envolvidos e considerando valores quantificáveis para as probabilidades desses riscos se tornarem problemas, apresenta-se abaixo uma estimativa para exposição ao risco.

\begin{table}[!h]
\begin{center}
\begin{tabular}{|p{3.5cm}|p{3.5cm}|p{4cm}|}
\hline \multicolumn{1}{|c|}{\T \B \textbf{Riscos}} & \multicolumn{1}{|c|}{\textbf{Probabilidade}} & \textbf{Estimativa = Probabilidade $\times$ Impacto}\\
\hline \T \B Mudança dos requisitos & Alta (valor 0,8) & $E(r) = 0,8 \times 8 = 6,4$\\
\hline \T \B Pressão excessiva do cronograma & Baixa (valor 0,1) & $E(r) = 0,1 \times 3 = 0,3$\\
\hline \T \B Requisitos problemáticos do utilizador & Alta (valor 0,8) & $E(r) = 0,8 \times 9 = 7,2$\\
\hline \T \B Qualidade baixa & Baixa (valor 0,1) & $E(r) = 0,1 \times 9 = 0,9$\\
\hline
\end{tabular}
 \caption{Tabela de Riscos}
\end{center}
\end{table}
\textbf{Impacto do risco}: Varia entre 1 e 10

\textbf{Probabilidade do risco}: Varia entre 0 e 1 (baixa, média e alta)
 
\section{Custos}
A duração global prevista para este projecto a ser efectuada por um recurso humano por dia é de seis meses. O custo dia de um recurso humano é de 50 euros. Sendo assim, o custo estimado para este projecto, considerando 21 dias úteis, é 12.600 euros. Assim serão admitidos ao projecto o número essencial de recursos humanos de forma a garantir a sua exequibilidade no prazo estipulado. 
Em relação à manutenção do sistema, será necessário contabilizar o custo de um administrador do sistema, pago de acordo com o valor de mercado.
Por fim acresce ao custo o valor da formação dada aos utilizadores do sistema. Esse custo é medido em função do número de horas praticadas.
Será ainda necessário comprar hardware no valor de 3.500 euros para suportar toda a estrutura do \textit{Delusion}.

\section{Documentação e formação do Utilizador}
\subsection{Documentação}
Neste projecto, fará parte a seguinte documentação:

\begin{itemize}
\item \textbf{Manual do utilizador} – Específica todas as funcionalidades do software, de forma simples e intuitiva, de maneira a ajudar os utilizadores na utilização do mesmo. Este manual será para uso de todos os utilizadores do sistema.

\item \textbf{Manual de administração} – Específica todas as tarefas necessárias para instalar e administrar o software. Refere também todos os recursos necessários ao nível de hardware e software de forma que o produto funcione sem problemas. Este documento será para uso exclusivo do administrador do sistema.
\end{itemize}

Esta documentação será elaborada por um subconjunto da equipa de desenvolvimento do projecto. A equipa será responsável também pela actualização da mesma. Toda a documentação será fornecida em formato digital (ficheiros pdf). 

\subsection{Formação dos utilizadores}
A formação dos utilizadores será numa primeira fase da responsabilidade de alguns dos engenheiros de software da equipa de desenvolvimento. Estes terão como responsabilidade ministrar acções de formação com os utilizadores do produto, de forma a instruir os utilizadores das principais funcionalidades do produto. Essas acções serão iniciadas após a instalação do produto. Estas acções de formação têm como finalidade envolver os utilizadores comuns do sistema no manuseamento do produto, de forma a torná-los autónomos da sua utilização.
