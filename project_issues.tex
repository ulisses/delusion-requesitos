\chapter{Questões do Projecto}
\minitoc
\section{Pontos em aberto}

\subsection{Especificação entidades sistema}
Da análise prévia já efectuada, foram identificados os seguintes intervenientes neste sistema:

\begin{itemize} 
\item Normal
\item Administrador
\end{itemize}

No entanto, é possível que com o desenvolver do projecto que novos tipos de utilizadores tenham de ser criados, de forma a corresponder com os previlégios de administração que uma determinada rede possui.
\section{Novos Problemas}

\subsection{Incorporação aplicações}

O ~\textit{Delusion} está a ser concebido de maneira a que seja possível através de um sistema de \emph{plugins} incorporar nos gráficos apresentados pelo \emph{visualizador} dados provenientes de ferramentas distintas.

\subsection{Manutenção do sistema}
O seguimento do sistema, depois da sua instalação, é assegurado por um serviço de suporte ao produto, disponibilizado pela empresa fornecedora do sistema. Contudo, esse suporte não cobre situações externas que possam influenciar o funcionamento deste, tais como problemas com infra-estrutura e comunicações, que terão que ser asseguradas pela empresa tomadora do sistema. 

\section{Custos}

A duração global prevista para este projecto a ser efectuada por um recurso humano por dia é de cinco meses. O custo dia de um recurso humano é de 50 euros. Sendo assim, o custo estimado para este projecto, considerando 21 dias úteis, é 75075 euros. Assim serão admitidos ao projecto o número essencial de recursos humanos de forma a garantir a sua exequibilidade no prazo estipulado. 
Em relação à manutenção do sistema, será necessário contabilizar o custo de um administrador do sistema, pago de acordo com o valor de mercado.
Por fim acresce ao custo o valor da formação dada aos utilizadores do sistema. Esse custo é medido em função do número de horas praticadas.
Será ainda necessário comprar hardware no valor de 3.500 euros para suportar toda a estrutura do \textit{Delusion}.

\section{Documentação e formação do Utilizador}
\subsection{Documentação}
Neste projecto, fará parte a seguinte documentação:

\begin{itemize}
\item \textbf{Manual do utilizador} – Específica todas as funcionalidades do software, de forma simples e intuitiva, de maneira a ajudar os utilizadores na utilização do mesmo. Este manual será para uso de todos os utilizadores do sistema.

\item \textbf{Manual de administração} – Específica todas as tarefas necessárias para instalar e administrar o \emph{software}. Refere também todos os recursos necessários ao nível de \emph{hardware} e \emph{software} de forma que o produto funcione sem problemas. Este documento será para uso exclusivo do administrador do sistema.
\end{itemize}

Esta documentação será elaborada por um subconjunto da equipa de desenvolvimento do projecto. A equipa será responsável também pela actualização da mesma. Toda a documentação será fornecida em formato digital (ficheiros pdf). 

\subsection{Formação dos utilizadores}
A formação dos utilizadores será numa primeira fase da responsabilidade de alguns dos engenheiros de \emph{software} da equipa de desenvolvimento. Estes terão como responsabilidade ministrar acções de formação com os utilizadores do produto, de forma a instruir os utilizadores das principais funcionalidades do produto. Essas acções serão iniciadas após a instalação do produto. Estas acções de formação têm como finalidade envolver os utilizadores comuns do sistema no manuseamento do produto, de forma a torná-los autónomos da sua utilização.
