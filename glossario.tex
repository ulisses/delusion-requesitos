\label{appendix:1}
\begin{description}
    \item \textbf{Autenticidade}

    Propriedade de segurança que tem como finalidade assegurar a “origem” da mensagem.
\end{description}

\begin{description}
    \item \textbf{Confidencialidade}

    Propriedade de segurança que tem como finalidade garantir que o conteúdo da mensagem só é do conhecimento dos intervenientes legítimos.
\end{description}

\begin{description}
    \item \textbf{Firewall}

    Uma firewall é um dispositivo ou um conjunto de dispositivos que tem como função permitir ou negar certas transmissões de rede, protegendo desta forma o sistema onde está instalada, pois todas as ligações não autorizadas são bloqueadas, permitindo apenas ligações legítimas.
\end{description}

\begin{description}
    \item \textbf{Gateway}
    
    Um gateway é um nodo de rede que actua como um elo de ligação entre duas redes distintas.
\end{description}

\begin{description}
    \item \textbf{Honeypot}

    Um Honeypot é um serviço que é instalado numa rede com os seguintes objectivos:
    \begin{itemize}
        \item desviar a atenção dos atacantes de serviços mais sensíveis e importantes;
        \item perceber os ataques infligidos.
    \end{itemize}
    
    Os dados recolhidos durante o funcionamento de um \textit{Honeypot} auxiliam os administradores de sistema a proteger os serviços reais.\\

    
\end{description}

\begin{description}
    \item \textbf{Integridade}

    Propriedade de segurança que tem com finalidade garantir que o receptor não “aceita” mensagens que tenham sido manipuladas.
\end{description}

\begin{description}
    \item \textbf{Intrusion Detection System}
    
    Um sistema de detecção de intrusões (\textit{IDS}) é um dispositivo que monitoriza a rede, ou até as actividades de um dado sistema, em busca de actividades maliciosas ou violações da política da rede imposta. O resultado desta monitorização é exportado através de \textit{logs}, para que mais tarda possa ser consultada pelos administradores de sistema.
\end{description}

\begin{description}
    \item \textbf{IpTables}
\end{description}

\begin{description}
    \item \textbf{Kernel}
\end{description}

\begin{description}
    \item \textbf{Máquina Virtual}
\end{description}

\begin{description}
    \item \textbf{Netflow}

O \textit{Netflow} é um protocolo da Cisco que colecciona informação de tráfego na rede. Esta ferramenta utiliza a nocção de \textit{network flow}, que consiste num conjunto de pacotes que partilham os mesmos endereços IP e portas de origem e destino, entre outros atributos, durante um determiado intervalo de tempo. 
\end{description}

\begin{description}
    \item \textbf{Nfcapd}

O \textit{Nfcapd}é uma ferramenta designada como um \textit{collector}, cujo objectivo é receber e agregar a informação de flows.
\end{description}

\begin{description}
    \item \textbf{Nmap}
\end{description}

\begin{description}
    \item \textbf{Qemu}
\end{description}

\begin{description}
    \item \textbf{Sniffer}
\end{description}

\begin{description}
    \item \textbf{Softflowd}

Esta ferramenta é designada como um \textit{exporter}, cujo objectivo é monitorizar a rede e exportar a informação de flows.
\end{description}

\begin{description}
    \item \textbf{Snort}

O Snort é um \textit{IDS (Intrusion Detection System) Open Source}. A sua função consiste em analisar o tráfego da rede em busca de possíveis ataques. Para isso, dispõe de uma base de dados de assinaturas de pacotes correspondentes a tráfego malicioso (chamadas regras). Esta base de dados é composta por regras criadas pelos utilizadores com finalidades diferentes, por exemplo: detectar IPs ou portas utilizadas, ou até análisar o payload dos pacotes.
\end{description}

\begin{description}
    \item \textbf{Router}
\end{description}

\begin{description}
    \item \textbf{Tabela de Encaminhamento}
\end{description}

\begin{description}
    \item \textbf{Tshark}
\end{description}

\begin{description}
    \item \textbf{WebService}
\end{description}
