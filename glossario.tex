\begin{description}
\item \textbf{Acidente}

O acontecimento fortuito, súbito e anormal, devido a causa exterior e violenta, estranha à vontade do Tomador de Seguro, do Beneficiário ou da Pessoa Segura, que origine lesões ou danos às pessoas ou bens seguros e que seja susceptível de fazer funcionar as garantias e coberturas do contrato de seguro.
\end{description}

\begin{description}
\item \textbf{Acta Adicional}

Documento que formaliza as alterações às condições gerais, especiais e particulares do contrato de seguro, deste fazendo parte integrante.
\end{description}

\begin{description}
\item \textbf{Actualização Automática de Capital}

Método utilizado para ajustar, periodicamente, o valor das coisas ou bens seguros aos efeitos da erosão monetária e da inflação. Para tal, tanto se podem utilizar índices publicados pelo Instituto de Seguros de Portugal (actualização indexada), como convencionar-se, previamente, uma percentagem de actualização (actualização convencionada).
\end{description}

\begin{description}
\item \textbf{Agente de Seguros}

Exerce a actividade de mediação de seguros em nome e por conta de uma ou mais empresas de seguros ou de outro mediador de seguros, nos termos do ou dos contratos que celebre com essa entidade.
\end{description}

\begin{description}
\item \textbf{Agravamento de Risco}

Consequência da verificação de circunstâncias, em momento posterior à celebração do contrato de seguro, que se mostrem susceptíveis de aumentar a intensidade ou a probabilidade da ocorrência de um risco e que, por isso, podem justificar o agravamento da taxa ou do prémio do seguro ou a alteração das condições de cobertura e, no limite, a sua recusa. A informação tempestiva de tais circunstâncias constitui obrigação do Tomador de Seguro, do Segurado ou da Pessoa Segura.
\end{description}

\begin{description}
\item \textbf{Agregado Familiar}

Conjunto de pessoas constituído pelo Segurado, o seu cônjuge ou pessoa com quem viva em união de facto, e os descendentes ou ascendentes que com eles vivam em comunhão de mesa e habitação e conforme conste expressamente das Condições Particulares do contrato.
\end{description}

\begin{description}
\item \textbf{Alienação}

Venda, troca, permuta, cessão e, em geral, qualquer transmissão a título oneroso, entre pessoas vivas, da propriedade ou de qualquer direito sobre determinado bem.
\end{description}

\begin{description}
\item \textbf{Alteração}

Modificação de condições, inicialmente contratadas, a fim de adaptar o contrato a novas circunstâncias. Se pedida pelo Tomador de Seguro, poderá ser aceite ou recusada pela Seguradora, com ou sem alteração do prémio; sendo aceite, será formalizada sob a forma de Acta Adicional.
\end{description}

\begin{description}
\item \textbf{Aniversário Contratual (Data de)}

Dia e mês do ano em que se celebra o início do Vencimento do contrato e que, caso o contrato tenha condições válidas para continuar a vigorar para "anos seguintes", será sempre tida como a data de início da Anuidade Contratual.
\end{description}

\begin{description}
\item \textbf{Anuidade Contratual}

Período de vigência anual de um contrato de seguro, que se inicia na sua data de Aniversário no corrente ano civil, até à véspera do mesmo dia e mês do ano civil seguinte.
\end{description}

\begin{description}
\item \textbf{Anulação (do Contrato)}

É uma das causas de invalidade de um contrato: mecanismo jurídico que permite pôr termo aos efeitos do contrato, mas apenas a partir da data em que o evento que a motivou ocorre e nunca retroactivamente, pelo que, em geral, a Seguradora deve restituir os prémios recebidos, calculados proporcionalmente ao período de tempo não decorrido até ao vencimento seguinte. É necessária a verificação de determinado motivo reconhecido, legal ou contratualmente, como justificativo dessa anulação e a qual deve ser comunicada por uma parte e aceite pela outra. Distingue-se da "Nulidade" pelo facto de esta ter efeitos retroactivos à data de início do contrato, tornando também nulos todos os efeitos que este poderia ter provocado, sem prejuízo de a Seguradora, em determinados casos, ter direito a reter os prémios pagos.
\end{description}

\begin{description}
\item \textbf{Apólice}

Documento escrito que, com base numa proposta de seguro, formaliza o contrato de seguro estabelecido entre o Tomador de Seguro ou o subscritor/Aderente e a Seguradora, nomeadamente fixando e regulando as suas condições de funcionamento. Dizem-se apólices uniformes aquelas cujas condições gerais, estipuladas pelo Instituto de Seguros de Portugal, são taxativa e obrigatoriamente seguidas por todas as Seguradoras. As apólices podem ser do tipo Individual, Frota ou Adesão. As apólices de Frota têm um só recibo e a Inclusão/Exclusão de coisas seguras e pessoas seguras são alterações às apólices. Cada apólice de Adesão corresponde a uma Pessoa Segura pertencente a uma apólice de grupo. Assim, cada apólice tem um recibo por adesão e a Inclusão/Exclusão de Pessoas Seguras é feita através de Adesões/Anulações às respectivas adesões da apólice de grupo.\end{description}

\begin{description}
\item \textbf{Apólice Recibo}

Documento simultaneamente proposta de seguro, apólice e prova de pagamento do recibo inicial ou único.
\end{description}

\begin{description}
\item \textbf{Arbitragem}

Meio extra-judicial de resolução de conflitos, previsto na legislação, segundo o qual as partes (Tomadores, Segurados, Beneficiários, Seguradoras, etc.) podem convencionar que qualquer litígio eventualmente decorrente (do contrato de Seguro) pode ser submetido à decisão de Árbitros (uma única ou um conjunto de pessoas singulares, reunidas em número impar - Tribunal albitral), os quais, respeitando os princípios legais, tomarão uma decisão, num prazo mais curto e determinado, que será notificada às partes e que, em princípio, será passível de Recurso para o Tribunal da Relação (nos mesmos termos que as sentenças dos Tribunais judiciais de comarca).
\end{description}

\begin{description}
\item \textbf{Beneficiário}

Pessoa singular ou colectiva, expressamente indicada pelo Tomador de Seguro ou pelo Aderente ou, na falta de indicação, nos termos previstos na Lei ou no contrato, a favor de quem reverterá a prestação da seguradora - indemnização ou entrega do capital - decorrente de um contrato de seguro ou de uma operação de capitalização.
\end{description}

\begin{description}
\item \textbf{Boletim de Adesão}

Documento pelo qual o candidato a pessoa segura declara desejar ser integrado no seguro de Grupo e que incluirá os dados individuais respectivos.
\end{description}

\begin{description}
\item \textbf{Bónus / Malus}

Sistema ou escala de bonificação ou agravamento do prémio em função da inexistência ou da ocorrência de sinistros participados ao abrigo da apólice.
\end{description}

\begin{description}
\item \textbf{Caducidade}

Extinção de um direito ou por decurso do prazo de vigência do mesmo, ou por se tornar ineficaz o seu exercício, em consequência de acontecimentos ou circunstâncias supervenientes ao início do contrato.
\end{description}

\begin{description}
\item \textbf{Capital}

Valor monetário correspondente ao valor máximo estabelecido para o objecto do contrato de seguro. Conforme os casos, este capital pode requerer designações mais específicas, tais como: - Capital (indemnização paga em): montante pago de uma só vez a uma vítima, em compensação de prejuízo sofrido (por oposição à noção de «renda»). - Capital seguro: valor ou montante declarado no contrato e constituindo o limite de responsabilidade do segurado, qualquer que seja a importância do dano. - Capital (em seguro de pessoas): montante pago ao beneficiário em função do risco coberto e do valor da garantia escolhida pelo segurado. - Capital constitutivo de renda: montante constituído como provisão, nas contas da seguradora, para poder garantir o pagamento de uma renda, até ao seu termo.\end{description}

\begin{description}
\item \textbf{Carta Verde}

Documento comprovativo da existência do seguro obrigatório de responsabilidade civil de automóvel, em termos válidos e eficazes, também designado por certificado internacional de seguro. A carta verde, enquanto prova da existência de seguro, é válida em todos os países nela mencionados.
\end{description}

\begin{description}
\item \textbf{Certificado De Tarifação}

Documento emitido pela seguradora, referente ao seguro de responsabilidade civil automóvel, que, no caso de resolução ou não renovação deste seguro, relata toda a experiência de um Segurado, quanto aos sinistros nos últimos 5 anos e ainda os agravamentos e bonificações do prémio em vigor, eventualmente a serem considerados por qualquer seguradora, em caso de celebração de novo contrato de seguro pela mesma pessoa e mesmo que para veículos diferentes.
\end{description}

\begin{description}
\item \textbf{Certificado Individual de Adesão}

Documento emitido pela seguradora, para cada uma das pessoas aderentes a seguros colectivos ou de grupo, e comprovativo da sua qualidade de "Pessoa segura".
\end{description}

\begin{description}
\item \textbf{Cliente}

É a pessoa, Individual ou Colectiva, Proponente, Tomador de Seguro, Aderente, etc., que contrata as condições contratuais com a Seguradora ou que é responsável pelo pagamento dos prémios de seguro.
\end{description}

\begin{description}
\item \textbf{Cobertura (ou Garantia) Contratual}

Conjunto de situações ou acontecimentos, previstos/garantidos no contrato, cuja verificação, nas circunstâncias nele definidas, dará lugar à prestação da seguradora.
\end{description}

\begin{description}
\item \textbf{Cobertura de Risco da Apólice}

Conjunto de riscos (de base e complementares ou especiais) subscritos através de uma apólice, que integram o produto de seguro contratado.
\end{description}

\begin{description}
\item \textbf{Co-existência de contratos}

Existência, em simultâneo, de dois ou mais contratos de seguro, em vigor no mesmo período de Vigência e cujo objecto seguro seja o mesmo bem, interesse ou valor, independentemente de quem os celebrou e da Seguradora que os aceitou.
\end{description}

\begin{description}
\item \textbf{Coisa Segura}

Bem, valor, objecto ou animal ou interesse sobre o qual se expressa a intenção de segurar.\end{description}

\begin{description}
\item \textbf{Comissão}

Remuneração pela angariação ou gestão de um contrato de seguro, de resseguro ou de retrocessão.\end{description}

\begin{description}
\item \textbf{Comissão de Cobrança}

Remuneração atribuída ao mediador em relação aos prémios de seguro por este efectivamente cobrados, desde que lhe tenham sido previamente atribuídas funções de cobrança pela empresa de seguros.
\end{description}

\begin{description}
\item \textbf{Condições Especiais}

Conjunto de cláusulas contratuais que completam e, nalguns casos esclarecem, as Condições Gerais e definem as Coberturas Complementares às Coberturas Base do contrato - contratáveis extraordinária e facultativamente pelo Tomador do Seguro -, bem como o respectivo âmbito, funcionamento e condições de contratação.
\end{description}

\begin{description}
\item \textbf{Condições Gerais}

Conjunto de cláusulas que definem e regulamentam os aspectos gerais de um contrato de seguro (duração, objecto, funcionamento das garantias, lei aplicável, direitos e obrigações das partes, etc.), comum a todos os contratos do mesmo ramo ou modalidade.
\end{description}

\begin{description}
\item \textbf{Condições Particulares}

Cláusulas do contrato de seguro que o individualizam, das quais constam, concreta e expressamente, a identificação e informações relativas ao Tomador de Seguro, ao segurado, às pessoa(s) segura(s) ou beneficiário(s), o objecto seguro, o capital seguro, o montante do prémio a pagar, o início e a duração do contrato, etc., bem como todas as condições especificamente contratadas (Condições Especiais) e as eventuais derrogações ou excepções às Condições Gerais do contrato.
\end{description}

\begin{description}
\item \textbf{Contrato de Seguro}

Contrato pelo qual uma seguradora, mediante o recebimento de determinada quantia - prémio ou prestação - se obriga a indemnizar outra entidade - segurado ou terceiro - pelos prejuízos sofridos, caso se verifique um risco ou, quando se trate de eventos relacionados com uma pessoa - vida, morte ou sobrevivência -, ao pagamento de um capital ou de uma renda.
\end{description}

\begin{description}
\item \textbf{Corretor de Seguros}

Exerce a actividade de mediação de seguros de forma independente face às empresas de seguros, baseando a sua actividade numa análise imparcial de um número suficiente de contratos de seguros disponíveis no mercado que lhe permita aconselhar o cliente tendo em conta as suas necessidades específicas.
\end{description}

\begin{description}
\item \textbf{Co-Seguro}

A assunção conjunta do risco de uma determinada apólice, por várias Seguradoras, que recebem a designação de Co-Seguradoras, de entre as quais uma é a leader (que tem a gestão da apólice), através de um único contrato de seguro, com as mesmas garantias e período de duração e com um prémio global.
\end{description}

\begin{description}
\item \textbf{Dano}

Prejuízo que deve ser reparado, indemnizado ou compensado. Pode ser patrimonial ou não patrimonial, directo ou indirecto, físico ou corporal, material ou moral.
\end{description}

\begin{description}
\item \textbf{Dano Corporal}

Lesão que afecta a saúde física ou mental.
\end{description}

\begin{description}
\item \textbf{Dano Material}

Lesão que afecte qualquer coisa móvel, imóvel, animal ou interesse jurídico.
\end{description}

\begin{description}
\item \textbf{Dano não Patrimonial}

Prejuízo que não sendo susceptível de avaliação pecuniária, deve, no entanto, ser compensado através de uma prestação pecuniária.
\end{description}

\begin{description}
\item \textbf{Dano Patrimonial}

Prejuízo que, sendo susceptível de avaliação pecuniária, deve ser reparado ou indemnizado.
\end{description}

\begin{description}
\item \textbf{Danos Próprios (Ramo Automóvel)}

Danos materiais causados ao veículo seguro em consequência de certos riscos cobertos e subscritos facultativa e complementarmente (choque, colisão ou capotamento, incêndio, raio ou explosão, furto ou roubo, cataclismos naturais e queda de aeronaves, greves, tumultos, comoções civis, vandalismo e actos de terrorismo).
\end{description}

\begin{description}
\item \textbf{Declaração Amigável de Acidente Automóvel}

Formulário-tipo (de modelo convencionado e reconhecido pelas Entidades oficiais e Seguradoras portuguesas), utilizado no relato das circunstâncias, para identificação dos veículos e condutores envolvidos num acidente automóvel e assinado por estes últimos. (Declaração válida para todos os tipos de acidente automóvel).
\end{description}

\begin{description}
\item \textbf{Denúncia}

Expressão frequentemente utilizada para significar o acto de notificação da rescisão e não-renovação do contrato (de seguro), para o seu próximo vencimento.
\end{description}

\begin{description}
\item \textbf{Depreciação}

Diminuição ou perda de valor de um bem devido à sua antiguidade, uso ou desgaste, podendo ser avaliada segundo critérios matemáticos ou financeiros.
\end{description}

\begin{description}
\item \textbf{Derrogação}

Revogação contratual, com vista a afastar, no todo ou em parte, a aplicação de condições inicialmente negociadas.
\end{description}

\begin{description}
\item \textbf{Doença}

A alteração involuntária do estado de saúde, não causada por acidente, com sintomatologia manifestada e passível de constatação médica objectiva.
\end{description}

\begin{description}
\item \textbf{Elegibilidade}

Susceptibilidade de ser considerado Pessoa Segura, num Grupo Seguro.
\end{description}

\begin{description}
\item \textbf{Empresa de Seguros (Seguradora)}

Entidade legalmente autorizada a exercer a actividade seguradora e que subscreve, com o Tomador, o contrato de seguro.
\end{description}

\begin{description}
\item \textbf{Encargos}

Importância acrescentada ao prémio puro de um seguro, referente a taxas fiscais ou para-fiscais, ou que serve para cobrir as despesas de aquisição, gestão e cobrança da seguradora, cujos limites são definidos e fixados pela lei ou por Norma Regulamentar do Instituto de Seguros de Portugal.\end{description}

\begin{description}
\item \textbf{Estorno}

Devolução ou reembolso ao tomador do seguro do prémio ou parte do prémio anterior e indevidamente pago.
\end{description}

\begin{description}
\item \textbf{Exclusão}

Circunstância ou acontecimento que, expressamente referidos como não estando cobertos pelo contrato de seguro, são insusceptíveis de desencadear a obrigação de indemnizar. As exclusões encontram-se previstas nas condições Gerais e Especiais do contrato e as eventuais derrogações à sua aplicação devem constar expressamente das Condições Particulares da apólice.
\end{description}

\begin{description}
\item \textbf{Exclusões Absolutas}

Situações ou acontecimentos que, nunca estão cobertos pelo contrato de seguro e que são sempre insusceptíveis de desencadear a obrigação de pagamento a cargo da seguradora.
\end{description}

\begin{description}
\item \textbf{Exclusões Relativas}

Situações ou acontecimentos que, apesar de inicialmente previstos como não cobertos pelo contrato de seguro, podem, mediante negociação expressa e em certas condições, ser susceptíveis de ficar a cargo da seguradora.
\end{description}

\begin{description}
\item \textbf{Franquia}

Parte do risco, expressa em valor, dias ou percentagem que, em caso de Sinistro, fica a cargo da Pessoa Segura e/ou do Tomador de Seguro, conforme o estabelecido nas Condições Particulares.
\end{description}

\begin{description}
\item \textbf{Franquia Absoluta}

Montante ou a percentagem sobre os prejuízos, ou Capital Seguro, definida nas condições particulares da apólice, que será suportada pelo segurado em todo e qualquer sinistro e que nunca pode ser repartido com a Seguradora.
\end{description}

\begin{description}
\item \textbf{Franquia Relativa}

Montante ou a percentagem sobre os prejuízos, ou Capital Seguro, definida nas condições particulares da apólice, que será suportada pelo segurado em todo e qualquer sinistro e que, mediante negociação expressa e em certas condições, pode ser repartido com a Seguradora.
\end{description}

\begin{description}
\item \textbf{Fundo de Acidentes de Trabalho (FAT)}

Organismo dotado de autonomia administrativa e financeira na dependência do Instituto de Seguros de Portugal e cuja principal função é assumir os encargos decorrentes da actualização das pensões devidas por acidentes de trabalho. O seu financiamento é sobretudo assegurado pelos Tomadores de seguro - uma percentagem sobre as remunerações seguras - e pelas Seguradoras - uma permilagem sobre as provisões matemáticas afectas às pensões.
\end{description}

\begin{description}
\item \textbf{Fundo Garantia Automóvel (FGA)}

Organismo que funciona junto do Instituto de Seguros de Portugal, ao qual compete satisfazer as indemnizações decorrentes de acidentes originados por veículos sujeitos ao seguro obrigatório e que sejam matriculados em Portugal ou em países terceiros em relação à União Europeia que não tenham gabinete nacional de seguros, ou cujo gabinete nacional de seguros não tenha aderido à Convenção Multilateral de Garantia entre Serviços Nacionais de Seguros. Quando o responsável seja desconhecido ou não beneficie de seguro válido ou eficaz, ou for declarada a falência da seguradora, o FGA garante indemnizações por morte ou lesões corporais. Quando o responsável for identificado, mas não beneficie de seguro válido ou eficaz, o FGA também pode garantir os danos materiais. O valor a pagar a este organismo corresponde a 2,5\% do prémio comercial do seguro automóvel.
\end{description}

\begin{description}
\item \textbf{Idade actuarial}

Idade do Segurado ou da Pessoa Segura, na data de início ou de renovação do contrato, arredondada para mais um ano, caso já faltem menos de 6 meses até à data do seu aniversário de nascimento.
\end{description}

\begin{description}
\item \textbf{IDS (Indemnização Directa ao Segurado)}

Convenção entre seguradoras destinada a facilitar e a acelerar a regularização de sinistros de responsabilidade civil automóvel. Para tal, uma aderente - credora - pode e deve, até ao limite convencionado, indemnizar o seu segurado por conta e na proporção da responsabilidade atribuída ao segurado da aderente responsável - devedora. Para que isto se verifique é necessário que: - Ambos os condutores preencham correctamente e assinem a Declaração Amigável de Acidente Automóvel (D.A.A.A.). - Intervenham apenas 2 veículos. - O acidente ocorra em território nacional. - Existam apenas danos materiais. - Os danos não excedam 1000 contos. (converter para euros) - Ambos os veículos tenham seguro válido nas companhias aderentes à Convenção.
\end{description}

\begin{description}
\item \textbf{Imposto De Selo}

Quantia legalmente devida e calculada em percentagem (entre 5\% a 9\%) do prémio comercial de um seguro.
\end{description}

\begin{description}
\item \textbf{Incontestabilidade (do seguro de vida)}

Impossibilidade de uma ou ambas as partes contratuais (Tomador de Seguro e/ou Seguradora) recusarem ou derrogarem cláusulas ou condições contratuais inicialmente negociadas.
\end{description}

\begin{description}
\item \textbf{Indemnização}

É a obrigação contratual da Seguradora de reparação de um dano ou prejuízo, através do pagamento do valor necessário à reposição da situação existente no momento anterior ao sinistro ou, se tal não for possível, à sua compensação por valor equivalente. Nos seguros Ramo «vida» não há lugar a indemnização propriamente dita, mas sim à entrega do valor contratado.
\end{description}

\begin{description}
\item \textbf{Instituto Nacional de Emergência Médica (INEM)}

Organismo público, de competências legalmente reconhecidas, a favor do qual reverte uma quantia legalmente estipulada e a suportar pelos Tomadores de certos ramos de seguros.
\end{description}

\begin{description}
\item \textbf{Invalidez}

Situação, clinicamente analisável, na qual se encontra a vítima, em consequência de um acidente ou doença, traduzida na incapacidade de realização de actos ou comportamentos - físicos ou inerentes às funções intelectuais -, normais e próprios da sua actividade pessoal ou profissional. A Incapacidade pode ser Absoluta, Parcial ou Total (quanto à gravidade) e/ou Permanente, Temporária ou Definitiva (quanto à durabilidade).
\end{description}

\begin{description}
\item \textbf{Lesão}

Ofensa ou dano que afecte a saúde física ou mental de alguém (lesão corporal) ou afecte qualquer bem, móvel ou imóvel, ou interesse jurídico (lesão material) causando um prejuízo.
\end{description}

\begin{description}
\item \textbf{Local de Risco}

Local onde se encontra ou situa a Coisa Segura.
\end{description}

\begin{description}
\item \textbf{Malus}

É o agravamento (por aumento) do montante do prémio de seguro, na renovação do contrato, verificadas certas circunstâncias, previstas contratualmente, designadamente a ocorrência de sinistro.
\end{description}

\begin{description}
\item \textbf{Mediador de Seguros}

Pessoa singular ou colectiva devidamente inscrita no Instituto de Seguros de Portugal e autorizada para apresentar, propor e preparar a celebração de contratos de seguros e prestar assistência. O mediador de seguros pode assumir a categoria de agente, de mediador de seguros ligado ou de corretor de seguros.
\end{description}

\begin{description}
\item \textbf{Mediador de Seguros Ligado}

Exerce a actividade da mediação de seguros em nome e por conta de uma empresa de seguros ou, com autorização desta, de várias empresas de seguros, desde que os produto que promova não sejam concorrentes, não recebendo prémios ou somas destinadas aos tomadores de seguros, segurados ou beneficiários e actuando sob inteira responsabilidade dessa ou dessas empresas de seguros, no que se refere à mediação dos respectivos produtos; em complemento da sua actividade profissional, sempre que o seguro seja acessório do bem ou serviço fornecido no âmbito da actividade principal, não recebendo prémios ou somas destinadas aos tomadores de seguros, segurados ou beneficiários e actuando sob inteira responsabilidade de uma ou várias empresas de seguros, no que se refere à mediação dos respectivos produtos.
\end{description}

\begin{description}
\item \textbf{Multi-Riscos}

Tipo de contrato de seguro que se caracteriza pela cobertura conjunta e simultânea de diversos riscos.
\end{description}

\begin{description}
\item \textbf{Nulidade (do Contrato)}

É outra das causas de Resolução do contrato de Seguro: mecanismo jurídico que permite a qualquer interessado, declarar (ou exigir judicialmente a declaração) a Invalidade do contrato, a todo o tempo e retroactivamente, logo que seja conhecida a verificação de determinado motivo (previsto na lei ou no contrato), justificativo dessa invalidade. A nulidade não depende de comunicação prévia, nem de aceitação pela outra parte. O facto de ter efeitos retroactivos à data de início de contrato, torna também nulos todos os efeitos que este poderia ter provocado, sem prejuízo de a Seguradora, em determinados casos, ter direito a reter os prémios pagos.
\end{description}

\begin{description}
\item \textbf{Obrigatórios (Seguros)}

Seguros cuja contratação é imposta legalmente, que têm como objectivo social a garantia da protecção das vítimas de determinados riscos ou a protecção de interesses socialmente relevantes.
\end{description}

\begin{description}
\item \textbf{Partes Comuns}

Um condomínio integra espaços de propriedade privada (as fracções autónomas) e outros de propriedade partilhada (as Partes Comuns). A lei define como Partes Comuns os seguintes espaços de um edifício: - Solo, alicerces, colunas, pilares, paredes-mestras e todas as partes restantes que constituem a estrutura do edifício; - Telhado (ou os terraços de cobertura, ainda que destinados ao uso específico de uma fracção); - Entradas, vestíbulos, escadas e corredores de uso ou passagem comum a dois ou mais condóminos; - Instalações gerais da água, electricidade, aquecimento, ar condicionado, gás, comunicações e semelhantes. Se o título constitutivo nada indicar em contrário, consideram-se ainda Partes Comuns: - Elevadores; - Pátios e jardins anexos ao edifício;  Dependências destinadas ao uso e habitação do porteiro; - Garagens; - Todos os espaços que não sejam definidos no título constitutivo como sendo de uso exclusivo de um dos condóminos.
\end{description}

\begin{description}
\item \textbf{Participação de Acidente}

Documento, a preencher e assinar pelo Segurado, com o objectivo de informar a seguradora da ocorrência de um evento susceptível de estar coberto (sinistro) pela apólice ou contrato, descrevendo e identificando as suas circunstâncias, dimensão dos danos, intervenientes, etc. Constitui uma obrigação legal e contratual do Segurado ou da Pessoa Segura.
\end{description}

\begin{description}
\item \textbf{Pensionista (de Seguros)}

Beneficiário legal de uma pensão devida por Acidente de Trabalho.
\end{description}

\begin{description}
\item \textbf{Período de Carência}

Espaço de tempo, fixado nas Condições Gerais e/ou Particulares, que medeia entre o início do Contrato ou a data de admissão, se posterior, e a data de entrada em vigor das coberturas garantidas pelas Apólices de Acidentes Pessoais, Doença, Saúde, etc.
\end{description}

\begin{description}
\item \textbf{Peritagem/Averiguação}

Actividade cujo objectivo principal consiste em estimar o valor dos danos, muitas vezes a par da análise (averiguação) das causas e circunstâncias de um sinistro.
\end{description}

\begin{description}
\item \textbf{Perito}

Profissional especializado, indicado pela seguradora ou pelo segurado, cuja função é a de avaliar os danos participados e analisar ou averiguar as causas e circunstâncias de um sinistro.
\end{description}

\begin{description}
\item \textbf{Pessoa Segura}

As pessoas, subscritoras ou membros dos respectivos Agregados Familiares cuja saúde ou integridade física se segura e identificadas nas Condições Particulares.
\end{description}

\begin{description}
\item \textbf{Prémio}

Preço do seguro. Pode ser comercial, bruto ou total.
\end{description}

\begin{description}
\item \textbf{Prémio Bruto}

Prémio comercial, acrescido das cargas relacionadas com a emissão do contrato, tais como o fraccionamento, custo de apólice, actas adicionais e certificado de seguro.
\end{description}

\begin{description}
\item \textbf{Prémio Comercial}

Custo teórico médio das coberturas do contrato, acrescido de outros custos, nomeadamente de aquisição e de administração do contrato, bem como de gestão e de cobrança.
\end{description}

\begin{description}
\item \textbf{Prémio Mínimo}

Parcela mínima do Prémio Total que, em caso de opção pelo pagamento fraccionado de prémios de seguro, a Seguradora aceita receber.
\end{description}

\begin{description}
\item \textbf{Prémio Total}

Prémio bruto acrescido das cargas fiscais e parafiscais e que corresponde ao preço final, a pagar pelo Tomador de Seguro à seguradora, pela contratação do seguro.
\end{description}

\begin{description}
\item \textbf{Proposta de Seguro}

É o documento pelo qual o candidato a tomador de seguro ou os candidatos a tomador e pessoa segura, se forem diferentes, declaram a sua intenção de subscrever o contrato de seguro, que contém os dados individuais respectivos, necessários à avaliação do risco pela seguradora. As respostas ou informações a constar de uma Proposta, devem ser prestadas com verdade, boa-fé e sem reticências ou dissimulações por parte dos declarantes, uma vez que comprometem as condições da aceitação, e já que a Proposta passará, posteriormente, a fazer parte integrante do contrato de seguro.
\end{description}

\begin{description}
\item \textbf{Pro-Rata Temporis}

Expressão latina, frequentemente utilizada na linguagem contratual, normalmente para definir o critério de cálculo de devolução da parte do prémio devido ao Tomador do seguro, em caso de o contrato cessar os seus efeitos antes da data inicialmente prevista para o fim da sua vigência. Com ela se pretende dizer que o valor de prémio a devolver é proporcional ao período de tempo pelo qual o contrato deixou de estar em vigor, tendo em consideração o prazo inicialmente contratado.
\end{description}

\begin{description}
\item \textbf{Quitação (Recibo de)}

Declaração assinada pelo beneficiário de uma indemnização, mediante a qual este se declara e comprova estar inteiramente ressarcido, desobrigando a seguradora, da prestação que originou esse pagamento, a partir desse momento.
\end{description}

\begin{description}
\item \textbf{Reclamação}

Apresentação de um pedido relativo a qualquer pretensão sobre direitos ou interesses legitimamente protegidos legal ou contratualmente.
\end{description}

\begin{description}
\item \textbf{Renda}

Valor periódico, temporário ou vitalício, imediato ou diferido, fixo ou variável, em que se traduz o pagamento de algumas indemnizações, aos Beneficiários, particularmente em contratos do Ramo Vida.
\end{description}

\begin{description}
\item \textbf{Renúncia}

Declaração voluntária de desistência ao exercício de um direito ou conjunto de direitos, que o renunciante afasta de si, sem atribuir ou ceder a outrem.
\end{description}

\begin{description}
\item \textbf{Resolução (ou Rescisão)}

É o mecanismo jurídico que permite a uma parte comunicar à outra a sua vontade de pôr termo ao contrato, seja na sequência da verificação de um motivo que a lei ou o contrato reconheçam como justificativo da resolução, ou sem necessidade de invocar um motivo (apenas no caso dos Tomadores de Seguro). Excepto se motivada por Nulidade, só produz efeitos para o futuro, pelo que, os efeitos produzidos antes do pedido da resolução não são afectados.
\end{description}

\begin{description}
\item \textbf{Responsabilidade Civil}

É a situação em que se encontra alguém  que, tendo praticado um acto ílicito (contrário à lei, ou à ordem pública), é obrigado a indemnizar o lesado dos prejuízos que lhe causou.
\end{description}

\begin{description}
\item \textbf{Responsabilidade Criminal ou Penal}

É a obrigação de uma pessoa singular se sujeitar à pena, que decorrer expressamente da respectiva condenação judicial, pela prática de um crime que se encontre consagrado na Lei Criminal/Penal.
\end{description}

\begin{description}
\item \textbf{Resseguro}

É o contrato pelo qual uma Seguradora - denominada Seguradora primitiva ou cedente -, mediante um determinado Prémio, (re-)segura, em parte ou na totalidade, os riscos que assumiu, junto de outra Empresa de Seguros - denominada Resseguradora ou cessionária. É o seguro do seguro e distingue-se do (Co-seguro), porque não existir relação ou contactos directos entre o Tomador de Seguros e a Resseguradora, já que o contrato de Resseguro é apenas estabelecido entre Empresas de Seguros.
\end{description}

\begin{description}
\item \textbf{Risco}

Evento incerto e de data incerta, estranho à vontade das partes contratantes, cuja materialização constitui o Sinistro e contra cuja ocorrência se pretende segurar.
\end{description}

\begin{description}
\item \textbf{Roubo}

É a prática do crime, tipificado e denominado como tal na Lei Penal, que consiste na apropriação, através de subtracção ou do constrangimento a que lhe seja entregue coisa móvel alheia, por meio de violência contra uma pessoa, de ameaça com perigo para a sua vida ou integridade física, ou pondo-a na impossibilidade de resistir.
\end{description}

\begin{description}
\item \textbf{S.N.B (Serviço Nacional de Bombeiros)}

Organismo público a favor do qual se cobra uma percentagem do prémio comercial dos ramos de Incêndio, Multi-riscos, Agrícola e Pecuário.
\end{description}

\begin{description}
\item \textbf{Salvados}

São os objectos salvos do Sinistro. No caso do Seguro Automóvel, é o Veículo danificado cujo custo de reparação é superior ao seu valor venal. Nesta circunstância, a indemnização deverá tomar por base o valor venal, mas o salvado será negociado pelo valor residual ou com o seu proprietário ou com terceiros.
\end{description}

\begin{description}
\item \textbf{Segurado}

A pessoa singular ou colectiva identificada nas Condições Particulares, que pode coincidir ou não com o Tomador de Seguro, e que é titular dos bens, valores, interesses ou obrigações que constituem o objecto do seguro.
\end{description}

\begin{description}
\item \textbf{Seguro}

É o conjunto de operações desenvolvidas pelas Empresas autorizadas para essa actividade (Seguradoras), com vista a poderem celebrar Contratos de Seguro e a obrigar-se ao cumprimento das prestações nestes incluídas.
\end{description}

\begin{description}
\item \textbf{Seguro de Grupo}

Seguro efectuado relativamente a um conjunto de pessoas ligadas entre si e ao Tomador de Seguro por um vínculo ou interesse comum, que não seja o interesse de segurar.
\end{description}

\begin{description}
\item \textbf{Seguro de Grupo Contributivo}

Seguro de grupo em que as Pessoas Seguras contribuem no todo ou em parte para o pagamento do Prémio.
\end{description}

\begin{description}
\item \textbf{Seguro de Grupo Não Contributivo}

Seguro de grupo em que o Tomador de Seguro contribui na totalidade para o pagamento do Prémio.
\end{description}

\begin{description}
\item \textbf{Seguro Individual}

Seguro efectuado relativamente a uma Pessoa Segura, que pode ou não coincidir com o Tomador de Seguro, ou a uma Pessoa Segura e seu Agregado Familiar.
\end{description}

\begin{description}
\item \textbf{Sinistro}

Qualquer acontecimento de carácter fortuito, súbito e imprevisto, susceptível de fazer funcionar as garantias do contrato.
\end{description}

\begin{description}
\item \textbf{Sobreprémio}

Aumento do prémio ou por agravamento do risco em relação ao risco normal ou referência ou por alargamento do âmbito da cobertura do contrato.
\end{description}

\begin{description}
\item \textbf{Sobre-Seguro}

Ou seguro excessivo: sempre que tenha sido declarado um capital superior ao correspondente ao valor do objecto seguro.
\end{description}

\begin{description}
\item \textbf{Sub-Rogação}

É a transmissão dos direitos do Segurado para a Seguradora, após a liquidação da mesma, para que ela possa, eventualmente, exigir ao responsável pelos danos, o reembolso do montante que houver despendido.
\end{description}

\begin{description}
\item \textbf{Subscrição}

Acto pelo qual a Seguradora assume a garantia de um risco.
\end{description}

\begin{description}
\item \textbf{Subscritor}

Entidade que celebra uma operação de capitalização com a Seguradora, sendo responsável pelo pagamento da prestação.
\end{description}

\begin{description}
\item \textbf{Sub-Seguro}

Ou Infra-Seguro: sempre que o valor seguro seja inferior ao valor do objecto seguro.
\end{description}

\begin{description}
\item \textbf{Tarifa}

Base de cálculo do Prémio Comercial.
\end{description}

\begin{description}
\item \textbf{Temporário (seguro)}

Seguro contratado por um prazo (datas de início e termo final) pré-determinado, em geral até ao máximo de um ano.
\end{description}

\begin{description}
\item \textbf{Terceiro}

É aquele que, em consequência de um sinistro coberto pelo contrato de seguro, sofra prejuízos susceptíveis de serem reparados ou indemnizados por força da lei ou do contrato de seguro.
\end{description}

\begin{description}
\item \textbf{Termo de um contrato ou adesão (data de)}

Data em terminam os efeitos de um contrato ou adesão e cessam as obrigações entre as partes contratuais.
\end{description}

\begin{description}
\item \textbf{Tomador do Seguro}

Pessoa, singular ou colectiva, que contrata com a Seguradora, sendo responsável pelo pagamento dos prémios.
\end{description}

\begin{description}
\item \textbf{Um Ano e Seguintes}

Seguro contratado sem limite de tempo pré-definido, automaticamente renovável em cada vencimento anual, salvo decisão em contrário (denúncia) das partes, comunicada com pré-aviso obrigatório.
\end{description}

\begin{description}
\item \textbf{Valor de Reconstrução}

Custo de reconstrução ou o valor matricial, no caso de expropriação ou demolição. Para o efeito, não é considerado o valor do solo ou do terreno.
\end{description}

\begin{description}
\item \textbf{Valor em Novo (Valor De Substituição)}

Convenção mediante a qual a Seguradora se obriga a indemnizar os bens seguros pelo seu valor de substituição em novo, mas até ao limite do capital garantido pela apólice.
\end{description}

\begin{description}
\item \textbf{Valor Venal}

Valor comercial de um bem, em condições normais de mercado, ou seja, valor pelo qual um objecto concreto pode ser normalmente vendido.
\end{description}

\begin{description}
\item \textbf{Vencimento (data de)}

Data de início da produção de efeitos e do cumprimento das obrigações contratuais. Nos termos da lei, é também a data em que os prémios, das Apólices de (um ano e seguintes), são devidos.
\end{description}

\begin{description}
\item \textbf{Vigência Contratual}

Período de validade e no qual se verificam os factos ou eventos geradores da produção de efeitos de um contrato.
\end{description}

