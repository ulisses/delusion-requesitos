\label{appendix:1}
\begin{description}
    \item \textbf{Administrador de Sistema}
    Um Administrador de sistema (também conhecido como Administrador de redes informáticas) é uma pessoa responsável pela manutenção de uma rede de informática, assegurando o bom funcionamento dos serviços disponibilizados. 
\end{description}

\begin{description}
    \item \textbf{Autenticidade}

    Propriedade de segurança que tem como finalidade assegurar a “origem” da mensagem.
\end{description}

\begin{description}
    \item \textbf{Confidencialidade}

    Propriedade de segurança que tem como finalidade garantir que o conteúdo da mensagem só é do conhecimento de intervenientes legítimos.
\end{description}

\begin{description}
    \item \textbf{Firewall}

    Uma firewall é um dispositivo, ou um conjunto de dispositivos, que tem como função permitir ou negar certas transmissões de rede, protegendo desta forma o sistema onde está instalada. Numa rede em que esteja instalada uma \textit{Firewall} todas as ligações não autorizadas são bloqueadas, permitindo apenas ligações legítimas.
\end{description}

\begin{description}
    \item \textbf{Gateway}
    
    Um gateway é um nodo de uma rede que actua como um elo de ligação entre duas redes distintas.
\end{description}

\begin{description}
    \item \textbf{Honeypot}

    Um Honeypot é um serviço que é instalado numa rede com os seguintes objectivos:
    \begin{itemize}
        \item desviar a atenção dos atacantes de serviços mais sensíveis e importantes;
        \item perceber os ataques infligidos.
    \end{itemize}
    
    Os dados recolhidos durante o funcionamento de um \textit{Honeypot} auxiliam os administradores de sistema a proteger os serviços reais.\\
    
    Geralmente num Honeypot são colocados serviços cujas vulnerabilidades são conhecidas, de forma a atrair os atacantes para uma máquina que não possui informações importantes.
\end{description}

\begin{description}
    \item \textbf{Integridade}

    Propriedade de segurança que tem com finalidade garantir que o receptor não “aceita” mensagens que tenham sido manipuladas.
\end{description}

\begin{description}
    \item \textbf{Intrusion Detection System}
    
    Um sistema de detecção de intrusões (\textit{IDS}) é um dispositivo que monitoriza a rede, ou até as actividades de um dado sistema, em busca de actividades maliciosas ou de violações da política da rede imposta. O resultado desta monitorização é exportado através de \textit{logs}, para que mais tarde possa ser consultada pelos administradores de sistema.
\end{description}

\begin{description}
    \item \textbf{Máquina Virtual}
    
    Com uma Máquina Virtual consegue-se simular através de software o funcionamento de um computador real. Este funcionamento permite correr sistemas operativos como se fossem aplicações de outros sistemas operativos.\\

    O sistema operativo onde a máquina virtual corre é designado de \textit{host}.
    O sistema operativo que está a ser virtualizado é designado de \textit{guest}.
\end{description}

\begin{description}
    \item \textbf{Netflow}

O \textit{Netflow} é um protocolo da Cisco que colecciona informação de tráfego na rede. Esta ferramenta utiliza a nocção de \textit{network flow}, que consiste num conjunto de pacotes que partilham os mesmos endereços IP e portas de origem e destino, entre outros atributos, durante um determinado intervalo de tempo. 
\end{description}

\begin{description}
    \item \textbf{Nfcapd}

O \textit{Nfcapd}é uma ferramenta designada como um \textit{collector}, cujo objectivo é receber e agregar a informação de flows.
\end{description}

\begin{description}
    \item \textbf{Nmap}

   O \textit{NMap} é uma ferramenta gratuita e \textit{open-source} usada para explorar uma infraestrutura de rede, ou até em auditorias de segurança. Com o \textit{Nmap} é possível encontrar que sistemas operativos estão instalados, que serviços -- incluindo versões --  estão disponibilizados nas várias máquinas da rede.
\end{description}

\begin{description}
    \item \textbf{Qemu}

    O \textit{QEMU} é uma ferramenta \textit{open-source} que permite criar máquinas virtuais.
\end{description}

\begin{description}
    \item \textbf{Sniffer}
    Um \textit{Sniffer} tem como finalidade interceptar e registar o trâfego gerado numa rede. 
\end{description}

\begin{description}
    \item \textbf{Softflowd}

    Esta ferramenta é designada como um \textit{exporter}, cujo objectivo é monitorizar a rede e exportar a informação de flows.
\end{description}

\begin{description}
    \item \textbf{Snort}

    O Snort é um \textit{IDS (Intrusion Detection System) Open Source}. A sua função consiste em analisar o tráfego da rede em busca de possíveis ataques. Para isso, dispõe de uma base de dados de assinaturas de pacotes correspondentes a tráfego malicioso (chamadas regras). Esta base de dados é composta por regras criadas pelos utilizadores, com finalidades diferentes, como por exemplo: detectar IPs ou portas utilizadas, ou até análisar o payload dos pacotes.
\end{description}

\begin{description}
    \item \textbf{System Call}

    Uma \textit{System Call} é um serviço disponibilizado pelo sistema operativo às aplicações, para que estas consigam utilizar os vários elementos do computador, como por exemplo escrever ou ler a partir de um ficheiro.
\end{description}

\begin{description}
    \item \textbf{Router}

    Um router é um dispositivo que permite a ligação entre duas ou mais redes distintas.
\end{description}

\begin{description}
    \item \textbf{Tshark}
    O \textit{Tshark} é uma ferramenta utilizada para observar na linha de comandos os registos gerados pelo \textit{Wireshark}.
\end{description}

\begin{description}
    \item \textbf{Wireshark}

    O \textit{Wireshark} é uma ferramenta utilizada para interceptar e registar pacotes de rede.
\end{description}

\begin{description}
    \item \textbf{Web Service}
    
    Um \textit{Web Service} é um método de comunição entre dois dispositivos electrónicos através de uma rede.
\end{description}
