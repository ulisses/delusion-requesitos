\begin{description}
    \item \textbf{Firewall}
\end{description}

\begin{description}
    \item \textbf{Gateway}
\end{description}

\begin{description}
    \item \textbf{Honeypot}
\end{description}

\begin{description}
    \item \textbf{Intrusion Detection System}
\end{description}

\begin{description}
    \item \textbf{IpTables}
\end{description}

\begin{description}
    \item \textbf{Kernel}
\end{description}

\begin{description}
    \item \textbf{Máquina Virtual}
\end{description}

\begin{description}
    \item \textbf{Netflow}

O \textit{Netflow} é um protocolo da Cisco que colecciona informação de tráfego na rede. Esta ferramenta utiliza a nocção de \textit{network flow}, que consiste num conjunto de pacotes que partilham os mesmos endereços IP e portas de origem e destino, entre outros atributos, durante um determiado intervalo de tempo. 
\end{description}

\begin{description}
    \item \textbf{Nfcapd}

O \textit{Nfcapd}é uma ferramenta designada como um \textit{collector}, cujo objectivo é receber e agregar a informação de flows.
\end{description}

\begin{description}
    \item \textbf{Qemu}
\end{description}

\begin{description}
    \item \textbf{Sniffer}
\end{description}

\begin{description}
    \item \textbf{Softflowd}

Esta ferramenta é designada como um \textit{exporter}, cujo objectivo é monitorizar a rede e exportar a informação de flows.
\end{description}

\begin{description}
    \item \textbf{Snort}

O Snort é um \textit{IDS (Intrusion Detection System) Open Source}. A sua função consiste em analisar o tráfego da rede em busca de possíveis ataques. Para isso, dispõe de uma base de dados de assinaturas de pacotes correspondentes a tráfego malicioso (chamadas regras). Esta base de dados é composta por regras criadas pelos utilizadores com finalidades diferentes, por exemplo: detectar IPs ou portas utilizadas, ou até análisar o payload dos pacotes.
\end{description}

\begin{description}
    \item \textbf{Router}
\end{description}

\begin{description}
    \item \textbf{Tabela de Encaminhamento}
\end{description}

\begin{description}
    \item \textbf{WebService}
\end{description}
